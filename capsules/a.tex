% This article is public domain.
\documentclass{ltugboat}

\usepackage{microtype}
\usepackage{graphicx}
\usepackage{ifpdf}
\usepackage[breaklinks,hidelinks,pdfa]{hyperref}

%%% Start of metadata %%%

\title{Example \TUB\ article}

% repeat info for each author.
\author{Karl Berry}
\EDITORnoaddress
\netaddress{https://tug.org/TUGboat}

%%% End of metadata %%%

\begin{document}

\maketitle

\begin{abstract}
We discuss the data and code for creating the online \TUB\ \HTML\ files
which are automatically generated, both the per-issue tables of contents
and the accumulated lists across all issues of authors, categories, and
titles. All the source files are available from
\url{https://tug.org/TUGboat}.
\end{abstract}

\section{Introduction}

\TUB\ has had web pages generated for the per-issue tables of contents
and the accumulated lists across all issues of authors, categories, and
titles since approximately 2005. David Walden and I worked on the
process together and wrote an article about it~\cite{Berry:TB32-1-23};
Dave wrote all of the code. More recently, we wanted to add some
features to the final output which necessitated writing a new
implementation. This short note describes that new work.



%\bibliographystyle{tugboat}% or plain if you don't have tugboat.bst
%\nocite{book-minimal}      % just making the bibliography non-empty
%\bibliography{xampl}       % xampl.bib comes with BibTeX

\makesignature
\end{document}

article:
- script not module.
- html toc reflects toc as printed (categories, ordering), plus covers
  and typo fixes etc.; lists*.html, otoh, have many unifications.
- basic brace-parsing via Text::Balanced.
- (much) cleanup of capsule files, enhancement of parsing.
- more unifications, add lists-regexps.txt to existing lists-*.
- page numbers key, much adding of \offset.
- lists-* not index-* because of bb.
- nytprof per line.
- weird perl /ee so can have regexps from file (thus "&$1acute;" with quotes).
- debug lines enough to see general flow, likely to have to cut down
  input (xtb09) and add much more to see.
- input files are tex; first translations, then regexps, then unifications.
- consistency check across all issues for difficulties, categories, no
   lists-* left unused; existing item.pdf; xxx and more?
- sort by [title/]volume/issue/page for stability and newest first.
- &ouml; in anchor name, even though allowed, simpler to omit; ditto t_ prefix.
- utility fns.
