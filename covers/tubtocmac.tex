% $Id$
% Public domain. Originally created by Karl Berry, 2017.
% Some common macros for cover4 and cover3.
% 
% These are not included in tugboat.front because
% cover3 does not need (or want) tugboat.front and tugboat.sty.

% do nothing if reading this file a second time.
\ifx\tubprimitiveinput\undefined
  \let\next\relax
\else
  \expandafter\let\expandafter\next\csname endinput\endcsname
\fi
\next

% Start by loading graphics so we can define \XeTeX.
% miniltx redefines \input, as does tugboat.sty, so save and restore.
\let\tubprimitiveinput = \input
\input miniltx
\input graphicx
\input url.sty
\let\input = \tubprimitiveinput

% make page numbers in toc be links? define for complete, undefined else.
%\let\tublinktoc=t % done in tbNNNcomplete.tex.

% Typeset page number #1, and make it a pdf link to the appropriate page
% \iftublinktoc is true. The argument is not just an integer, but rather
% something using \getfirstpage or \getlastpage (from tugboat.dates).
% Those macros in turn use \input, etc.
% 
% So the whole thing is complicated, but necessary, because we don't
% want to ever write any hardwired page numbers -- they will inevitably
% become out of date at the last second, and it would be hard to notice.
% 
\newcount\tublinkdestpage
%
\def\tubmaybelinkpageno#1{%
  \ifx\tublinktoc\undefined \else
    {%
     % The \getfirstpage defined in tugboat.dates is fine for
     % typesetting, but for linking we need to get the number, not
     % typeset it. So redefine it. Similarly \getlastpage.
     %
      \tublinkdestpage = -1
      %
      \def\getfirstpage##1{\begingroup\globaldefs=1
        \def\tubfirstpage{\tublinkdestpage}%
        \input ../##1/firstpage.tex
      \endgroup}%
      %
      \def\getlastpage##1{\begingroup\globaldefs=1
        \def\tublastpageplus{\tublinkdestpage}%
        \input ../##1/lastpage.tex
        \advance\tublastpageplus by -1 % don't want "plus"
      \endgroup}%
      %
      % execute code to store the page number in \tublinkdestpage.
      \setbox0=\hbox{#1}%
      %
      \global\advance\tublinkdestpage by -\TBtitlepage % from tugboat.dates
      \global\advance\tublinkdestpage by 3 % cover4,cover2,titlepage
    }%
  \fi
  \tubmaybelinktext{#1}% typeset the text, make the link
}

% Link text #1 to \tublinkdestpage as set above.
% It turns out that we typeset the page number before anything else, 
% so no need to redo the calculations.
% 
\def\tubmaybelinktext#1{%
  % do not make link unless requested (set in toclinks.tex).
  \ifx\tublinktoc\undefined \else
    % do not make link if page number is nonexistent.
    \ifnum\tublinkdestpage > 0
      %\message{linking text to \the\tublinkdestpage}%
      \pdfstartlink
        attr{/Border [0 0 0]}    % boxes around links are too ugly
        goto page \tublinkdestpage{/XYZ}\relax  % destination page
    \fi
  \fi
  {#1}% Typeset text.
  \ifx\tublinktoc\undefined\else \ifnum\tublinkdestpage>0 \pdfendlink \fi \fi
}

%  all for xetex.
\def\tubreflect#1{%
  \ifx\reflectbox\undefined
    \errmessage{graphics package must be loaded}%
  \else
    \ifdim \fontdimen1\font>0pt
      \raise 1.75ex \hbox{\kern.1em\rotatebox{180}{#1}}\kern-.1em
    \else
      \reflectbox{#1}%
    \fi
  \fi
}
\def\tubhideheight#1{\setbox0=\hbox{#1}\ht0=0pt \dp0=0pt \box0 }
\def\XekernbeforeE{-.125em}
\def\XekernafterE{-.1667em}
\DeclareRobustCommand\Xe{\leavevmode
  \tubhideheight{\hbox{X%
    \setbox0=\hbox{\TeX}\setbox1=\hbox{E}%
    \ifdim \fontdimen1\font>0pt
      % XeTeX logo needs tinkering when slanted/italic font.
      \def\XekernbeforeE{-.11em}%
      \def\XekernafterE{-.11em}%
      \dp1=-.17ex
    \fi
    \lower\dp0\hbox{\raise\dp1\hbox{\kern\XekernbeforeE\tubreflect{E}}}%
    \kern\XekernafterE}}}
\def\XeTeX{\Xe\TeX}
\def\XeLaTeX{\Xe{\kern.11em \LaTeX}}

% from gm, improves \AllTeX spacing:
\def\La{\TestCount=\the\fam \leavevmode L%
        \setbox\TestBox=\hbox{$\fam\TestCount\scriptstyle A$}%
        \kern
          -.57 %\ifdim \fontdimen1\font>0pt -.4 \else -.57 \fi
          \wd\TestBox
        \raise
          \ifdim \fontdimen1\font>0pt .45ex \else .42ex \fi
          \copy\TestBox
        \kern
          \ifnum \TestCount=\itfam -.29 \else -.2 \fi
          \wd\TestBox
        }
\def\LaTeX{\La\TeX}

% extra logos.
\def\XyM{X\kern-.30em\smash{%
\raise.50ex\hbox to0.8em{\hss
{%\expandafter\csname OT1/cmr/\f@series/\f@shape/\f@size\endcsname
 \char'7 }%
\hss}}\kern-.30em{M}}
\def\XyMTeX{\XyM\kern-.1em\TeX}
\def\OpTeX{Op\kern-.05em\TeX}
\def\BibLaTeX{\Bib\kern.02em \LaTeX}

% prefer rm "and" with sl names.
\def\aand{\unskip{\rm\ and\ }\ignorespaces} % matches tub, can be too long

% usual backslash prefix
\def\cs#1{{\tt \char`\\#1}}

\def\code#1{\hbox{\tt #1}}
\let\pkg=\code

\def\titleref#1{{\sl\frenchspacing#1\/}}
\let\booktitle=\titleref % older name

% improve \Thanh
\def\Thanh{H\`an Th{\thanhfont \char'270} Th\`anh}

% improve Lslash/lslash
\def\L{{\lslashfont \char'212 }}
\def\l{{\lslashfont \char'252 }}
%
\def\Lrm{{\lslashrmfont \char'212 }}
\def\lrm{{\lslashrmfont \char'252 }}

% make \ldots work in text mode.
\let\plainldots=\ldots
\def\ldots{\relax\ifmmode\plainldots\else$\plainldots\,$\fi}
\let\dots=\ldots

% fancy plus for ``Grail+'', tb45:1 = tb139, like C++.
\def\tubplus{\raise.66ex\hbox{$_{+}$}}
%\def\tubplus{% LaTeX version:
%  \texorpdfstring
%     % switch to bold if that's the current NFSS "series".
%    {\raisebox{.7ex}{\ifx\f@series\bfseries@rm\boldmath\fi $_{+}$}}
%    {+}} 

% some fake LaTeX
\def\emph#1{{\it #1}}
\def\textbf#1{{\bf #1}}
\def\textit#1{{\it \def\&{\sl\char`& }#1}}
\def\textrm#1{{\rm #1}}
%\def\textsf#1{{\sf #1}} % intentionally leave out, we want to use tt
\def\textsl#1{{\sl #1}}
\def\texttt#1{{\tt #1}}

% | used for cmr in practice, so math bar is ok.
% we are presumably not in math mode already?
% \ensuremath from LaTeX is a bit too much for right now.
\def\textbar{\ifmmode\vert\else$\vert$\fi}

% slanted \BibTeX and \MF:
\def\slBibTeX{{\sl B{\slc IB}\kern-.08em\TeX}}
\def\BibTeX{\slBibTeX}
\def\Bib{B\kern-.05em{\smc i\kern-.025em b}}
% \manualsl is defined in cover?.tex, since might change in a given issue.
\def\slMF{{\manualsl META}\-{\manualsl FONT}\spacefactor1000 }
\def\slMP{{\manualsl META}\-{\manualsl POST}\spacefactor1000 }

\frenchspacing
\hyphenpenalty = 10000
\exhyphenpenalty = 10000
