% $Id$
% Public domain.
% 
% Test document for a first column of an article being all text,
% with no section headings. The hope is that it will not be underfull or
% overfull; see \BelowTitleSkip in tugboat.dtx.
% 
\documentclass[final,runningminimal]{ltugboat}
\usepackage[T1]{fontenc}
\usepackage{graphicx}
\usepackage{microtype}
\usepackage[hidelinks,pdfa]{hyperref}

\title{Example \TUB\ article}

% repeat info for each author; comment out items that don't apply.
\author{First Last}
\address{Street Address \\ Town, Postal \\ Country}
\netaddress{user (at) example dot org}
\personalURL{https://example.org/~user/}

\begin{document}
\maketitle

\def\lyearnn{00}\def\lyear{19\lyearnn}
\def\fyearnn{01}\def\fyear{19\fyearnn}
\def\nyearnn{02}\def\nyear{19\nyearnn}

The financial statements for \fyear\ have been reviewed by the \TUG{}
board but have not been audited. The totals may vary slightly due to
rounding. As a \acro{US} tax-exempt organization, \TUG's annual
information returns are publicly available on our web site, below.
%\url{https://tug.org/tax-exempt}.

Membership dues revenue was slightly down in \fyear\ compared to
\lyear{}; we ended the year with 1,162 paid members, 12 fewer than in
\lyear. The \fyear\ online conference had a small loss, due mostly to
unfavorable exchange rate variations. General contributions and product
sales returned to their normal levels after last year's one-time large
contributions. Thus, \fyear\ income was down around 33\%.

\TUB\ production and mailing fees increased substantially. With that and
one-time costs incurred with a change in the office, our bottom line for
\fyear{} was negative: $-$\$26,167.

\TUG's end-of-year asset total decreased, following that loss.

Committed Funds are reserved for designated projects: \LaTeX, \CTAN,
Mac\TeX, the \TeX\ development fund, and others
(\tbsurl{https://tug.org/donate}).
Incoming donations are allocated accordingly and disbursed as the
projects progress.
\TUG\ charges no overhead to administer these funds.

The Prepaid Member Income category is member dues that were paid in
earlier years for the current year (and beyond). The \fyear{} portion of
this liability was converted into regular Membership Dues in January of
\fyear{}. The payroll liabilities are for \fyear\ state and federal
taxes due in January, \nyear{}.

We have increased membership fees slightly in \nyear, for the first time
in many years, as inflation and shipping costs have not stood still.
Worldwide support from members and donations are what allow us to
continue, so thank you! As always, we welcome ideas to attract new
members.

We continue to provide members with a monthly newsletter, journals and
software, online support lists, and efficient service.

The Prepaid Member Income category is member dues that were paid in
earlier years for the current year (and beyond). The \fyear{} portion of
this liability was converted into regular Membership Dues in January of
\fyear{}. The payroll liabilities are for \fyear\ state and federal
taxes due in January, \nyear{}.

The Prepaid Member Income category is member dues that were paid in
earlier years for the current year (and beyond). The \fyear{} portion of
this liability was converted into regular Membership Dues in

\loggingall
\end{document}
