% \iffalse (this is a meta-comment (so they say))
% $Id$
% tugboat.dtx - main source for LaTeX TUGboat classes.
%
% Copyright 1994-1998, 2000-2001, 2004-2012, 2013 TeX Users Group.
%
% This file is part of the tugboat package.
%
% This work may be distributed and/or modified under the
% conditions of the LaTeX Project Public License, either version 1.3
% of this license or (at your option) any later version.
% The latest version of this license is in
%   http://www.latex-project.org/lppl.txt
% and version 1.3 or later is part of all distributions of LaTeX
% version 2003/12/01 or later.
%
% \fi
%
% \section{Document preambles}
%
%    \begin{macrocode}
%<ltugboatcls|ltugproccls|ltugcomn>\NeedsTeXFormat{LaTeX2e}[1994/12/01]
%<*dtx>
\ProvidesFile                 {tugboat.dtx}
%</dtx>
%<ltugboatcls>\ProvidesClass  {ltugboat}
%<ltugproccls>\ProvidesClass  {ltugproc}
%<ltugboatsty>\ProvidesPackage{ltugboat}
%<ltugprocsty>\ProvidesPackage{ltugproc}
%<ltugcomn>   \ProvidesPackage{ltugcomn}
                   [2013/01/31 v2.12
%<ltugboatcls>                       TUGboat journal class%
%<ltugproccls>                       TUG conference proceedings class%
%<ltugboatsty|ltugprocsty>           TUG compatibility package%
%<ltugcomn>                          TUGboat `common macros' package%
%<*dtx>
                                     TUG macros source file%
%</dtx>
                   ]
%    \end{macrocode}
%
% ^^A behold, a horrible hack...
%    \begin{macrocode}
%<*dtx>
\newif\ifoldlongtable
%</dtx>
%    \end{macrocode}
%
% \iffalse
%
% Note that the version control $Id above relates to the number of times
% the file has been checked-in, which relates to a development process
% taking place on several machines, rather than the incidence of
% significant change; the definitive version number is that in the
% \Provides* things, while the $Id may provide a clue as to the latest
% date the documentation was worked on
%
% Code to compile the thing using \LaTeXe.
%
%    \begin{macrocode}
%<*driver>
\documentclass{ltxdoc}
%
\usepackage{ltugcomn} % generated using the .ins file
%
\usepackage{array,longtable}
%
% if longtable.sty is the most recent, then it doesn't need
% \cs{setlongtables}, so we avoid the setlong.tab game
\GetFileInfo{longtable.sty}
\def\extractdate#1/#2/#3\relax{\def\fileyear{#1}%
  \def\filemonth{#2}%
  \def\fileday{#3}}
\expandafter\extractdate\filedate\relax
\expandafter\ifnum\fileyear>1996\relax
  \oldlongtablefalse
\else
  \oldlongtabletrue
  \expandafter\ifnum\fileyear=1996\relax
    \expandafter\ifnum\filemonth>5\relax
      \oldlongtablefalse
    \else
      \expandafter\ifnum\filemonth=5\relax
        \expandafter\ifnum\fileday>23\relax
          \oldlongtablefalse
        \fi
      \fi
    \fi
  \fi
\fi
\ifoldlongtable
  \InputIfFileExists{setlong.tab}{}% if this file exists, it should
%                                  % contain (only) \setlongtables
    {\typeout{*** create a file `setlong.tab' containing only}
     \typeout{***\space\space\space\string\setlongtables}
     \typeout{*** to get the tables of commands, etc., lined up}}
\fi
%
\begin{document}
 \GetFileInfo{tugboat.dtx}
 \title{The \textsf{tugboat} package\thanks{This file
        has version number \fileversion, last
        revised \filedate}}
    \author{The \TUB{} team\\
            \small(Distributed by Robin Fairbairns)}
 \date{\filedate}
 \maketitle
 \tableofcontents
 \DocInput{tugboat.dtx}
\end{document}
%</driver>
%    \end{macrocode}
%
% \fi
%
% \section{Introduction}
%
% This file contains all the macros for typesetting \TUB{} with
% both plain \TeX\ and \LaTeXe.
% \subsection{Summary of control sequences}
%
% Abbreviations.  Just a listing with indications of expansion where
% that may not be obvious.  For full definitions, see real code below
% (Section~\ref{abbr-logo}).
%
% \begin{longtable}{lp{3in}}
% |\AllTeX|&        (\La)\TeX\\
% |\AMS|&           American Mathematical Society\\
% |\AmSTeX|&\\
% |\aw|&            \aw\ (abbreviation for Addison-Wesley)\\
% |\API|&\\
% |\AW|&            Addison-Wesley\\
% |\BibTeX|&\\
% |\CandT|&         Computers \& Typesetting\\
% |\ConTeXt|&       \ConTeXt\\
% |\Cplusplus|&     C++\\
% |\DTD|&\\
% |\DVD|&\\
% |\DVI|&\\
% |\DVIPDFMx|&      \DVIPDFMx\\
% |\DVItoVDU|&      DVItoVDU\\
% |\ECMA|&\\
% |\EPS|&\\
% |\eTeX|&          \eTeX\\
% |\ExTeX|&         \ExTeX\\
% |\Ghostscript|&\\
% |\Hawaii|&        \Hawaii\\
% |\HTML|&\\
% |\ISBN|&          ISBN\\
% |\ISO|&\\
% |\ISSN|&          ISSN\\
% |\JTeX|&\\
% |\JoT|&           The Joy of \TeX\\
% |\LaTeX|&\\
% |\LyX|&\\
% |\MacOSX|&        Mac OS X\\
% |\MathML|&\\
% |\Mc|&            M with raised c\\
% |\MF|&            \MF\\
% |\mf|&            \textsc{Metafont}\\
% |\MFB|&           The Metafont book\\
% |\MP|&            \MP\\
% |\mp|&            MetaPost (in text only: remains `\ensuremath{\mp}'
%                   in maths)\\
% |\OMEGA|&         Omega `logo' (\ensuremath{\Omega})\\
% |\OCP|&           Omega compiled process\\
% |\OOXML|&\\
% |\OTP|&           Omega translation process\\
% |\mtex|&          multilingual \TeX\\
% |\NTS|&           New Typesetting System\\
% |\pcMF|&          pcMF\\
% |\PCTeX|&\\
% |\pcTeX|&\\
% |\Pas|&           Pascal\\
% |\PiCTeX|&\\
% |\plain|&         plain (in typewriter font)\\
% |\POBox|&         P. O. Box\\
% |\PS|&            PostScript (with hyphenation)\\
% |\SC|&            Steering Committee\\
% |\SGML|&          SGML\\
% |\SliTeX|&\\
% |\slMF|&          Metafont (slanted)\Dash deprecated: use
%                   \cs{textsl} instead \\
% |\stTeX|&         \TeX{} for the Atari ST\\
% |\SVG|&\\
% |\TANGLE|&\\
% |\TB|&            The \TeX book\\
% |\TeX|&           (Although nearly every package defines this,
%                   most\Dash including plain\Dash are missing the
%                   spacefactor adjustment)\\
% |\TeXhax|&\\
% |\TeXMaG|&        (defunct)\\
% |\TeXtures|&\\
% |\TeXXeT|&\\
% |\Thanh|&\\
% |\TFM|&           \TFM\\
% |\TUB|&           \TUB\\
% |\TUG|&           \TeX{} Users Group\\
% |\UNIX|&\\
% |\UTF|&\\
% |\VAX|&\\
% |\VnTeX|&\\
% |\VorTeX|&\\
% |\XeT|&\\
% |\XeTeX|&         reflected and lowered first `E'\\
% |\XeLaTeX|&       with extra space before `L'\\
% |\XML|&\\
% |\WEB|&\\
% |\WEAVE|&\\
% |\WYSIWYG|&\\
% \end{longtable}
%
% Macros for things that are slightly more significant.
%
% \begin{longtable}{lp{.6\textwidth}}
% |\NoBlackBoxes|&   turns off marginal rules marking overfull boxes\\
% |\BlackBoxes|&  turns them back on\\
% |\newline|&    horizontal glue plus a break\\
% |\ifundefined#1|&   checks argument with |\csname| against |\relax|\\
% |\topsmash|&   smashes above baseline (from AMSTeX)\\
% |\botsmash|&   smashes below baseline (from AMSTeX)\\
% |\smash|&   smashes both  (from plain)\\
% |\ulap|&   lap upwards\\
% |\dlap|&   lap downwards\\
% |\xlap|&   reference point at center horizontally; 0 width\\
% |\ylap|&   reference point at center vertically; 0 height, depth\\
% |\zlap|&   combination |\xlap| and |\ylap|\\
% |\basezero|&   to avoid insertion of baselineskip and lineskip glue\\
% |\nullhrule|&   empty |\hrule|\\
% |\nullvrule|&   empty |\vrule|\\
% |\makestrut[#1;#2]|& ad hoc struts; |#1|=height, |#2|=depth\\
% |\today|&   today's date\\
% |\SetTime|&    converts |\time| to hours, minutes\\
% |\now|&    displays time in hours and minutes\\
% |\Now|&    shows current date and time\\
% |\ifPrelimDraft|&   flag to indicate status as preliminary draft\\
% |\rtitlex|&    \TUB{} volume and number info for running head\\
% |\midrtitle|&   information for center of running head\\
% |\HorzR@gisterRule|& pieces of registration marks (`trimmarks')\\
% |\DownShortR@gisterRule|&\\
% |\UpShortR@gisterRule|&\\
% |\ttopregister|&   top registration line with `T' in center\\
% |\tbotregister|&   bottom registration line with inverted `T' in center\\
% |\topregister|&  register actually used\\
% |\botregister|&\\
% |\raggedskip|&  parameters used for ragged settings\\
% |\raggedstretch|&\\
% |\raggedparfill|&\\
% |\raggedspaces|&\\
% |\raggedright|&\\
% |\raggedleft|&\\
% |\raggedcenter|&\\
% |\normalspaces|&\\
% |\raggedbottom|&\\
% |\bull|&       square bullet\\
% |\cents|&      `cents' sign\\
% |\Dag|&        superscripted dagger\\
% |\careof|&     c/o\\
% |\sfrac|&      slashed fraction (arguments optionally\\
% &              separated by a slash)\\
% |\cs|&         control seqence name\\
%      &         |\cs{name}|$\rightarrow$\cs{name}\\
% |\env|&        environment name \\
%       &        |\env{name}|$\rightarrow$\env{name}\\
% |\meta|&       meta-argument name\\
%        &       |\meta{name}|$\rightarrow$\meta{name}\\
% |\dash|&       en-dash surrounded by thinspaces; only breakable AFTER\\
% |\Dash|&       em-dash, as above\\
% |\hyph|&       permit automatic hyphenation after an actual hyphen\\
% |\slash|&      `breakable' slash\\
% |\nth|&        for obtaining `1\textsuperscript{st}',
%                `2\textsuperscript{nd}', 3\textsuperscript{rd},
%                etc.\\
% |\tubissue|&   gets |\TUB| followed by volume and issue numbers\\
% |\xEdNote|&    Editor's Note:\\
% |\Review:|&    Review: (for title of book review article)\\
% |\reviewitem|& begin data for item being reviewed\\
% |\revauth|&    with one argument, author(s) of item being reviewed\\
% |\revtitle|&   with one argument, title of \ldots \\
% |\revpubinfo|& with one argument, other info pertaining to \ldots \\
% |\endreviewitem|& end data for item being reviewed\\
% |\booktitle|&  with one argument, format book title in text\\
% |\Input|&      |\input| with some other bookkeeping for
%                case where multiple articles are put together\\
% |\TBremark|&   reminder to \TUB{} editorial staff\\
% |\TBEnableRemarks|& enable \cs{TBremark}s (normally suppressed)\\
% |\pagexref|&   used to write out page numbers to screen and\\
% |\pagexrefON|& external files\\
% |\pagexrefOFF|&\\
% |\xrefto|&     used for symbolic cross-reference to other pages\\
% |\xreftoON|&   in \TUB{}\\
% |\xreftoOFF|&\\
% |\TBdriver|&   marks code which only takes effect when articles
%               are run together in a driver file\\
% \iffalse % these two were defined but never used! -- remove them
%          % from the documentation, at least (further research needed)
% |\signatureat|&  \\
% |\signaturepercent|&\\
% \fi
% |\signaturemark|& items for signatures\\
% |\signaturewidth|&\\
% \end{longtable}
%
% \StopEventually{}
%
% \section{\LaTeXe{} \TUB{} class file}
%
% \subsection{Setup and options}
%
% Check for reloading.  Hmmm\dots  Does this happen with \LaTeXe\
% classes?  Probably, in fact, as well that it doesn't, since the
% \cs{tugstyinit} referenced here doesn't exist; however, it's
% possible that we might need a similar mechanism in the future, so we
% retain its skeleton, without fleshing out the \cs{tugstyinit} bones.
%
%    \begin{macrocode}
%<*ltugboatcls>
\csname tugstyloaded@\endcsname
\def\tugstyloaded@{\tugstyinit\endinput}
%    \end{macrocode}
%
% Acquire a name for this class if we don't already have one (by
% virtue of having been loaded by |tugproc.cls|).  This name will be
% used in error messages and the like.
%
%    \begin{macrocode}
\providecommand{\@tugclass}{ltugboat}
%    \end{macrocode}
%
% Warnings/error messages/information messages~--- if we're using
% \LaTeXe{} we can use the \cs{Class*} commands:
%
%    \begin{macrocode}
\def\TBInfo{\ClassInfo{\@tugclass}}
\def\TBError{\ClassError{\@tugclass}}
\def\TBWarning{\ClassWarning{\@tugclass}}
\def\TBWarningNL{\ClassWarningNoLine{\@tugclass}}
%    \end{macrocode}
%
% Some trivial options, just flicking switches, etc.
%
%    \begin{macrocode}
\newif\ifpreprint
\def\preprint{\preprinttrue}
\DeclareOption{draft}{%
  \AtEndOfClass{%
    \setcounter{page}{1001}%
    \BlackBoxes
    \def\MakeRegistrationMarks{}%
    \PrelimDrafttrue
  }%
}
\DeclareOption{preprint}{%
   \preprinttrue
}
\DeclareOption{final}{%
  \AtEndOfClass{%
    \NoBlackBoxes
    \PrelimDraftfalse
    \@tubrunningfull
    }%
}
%    \end{macrocode}
%
% The rules dictate that the output should be set using a 10pt base
% font.
%
%    \begin{macrocode}
\DeclareOption{11pt}{%
  \TBWarning{The \@tugclass\space class only supports 10pt fonts:
    \MessageBreak option \CurrentOption\space ignored}%
}
\DeclareOption{12pt}{\csname ds@11pt\endcsname}
%    \end{macrocode}
%
% Similarly, ignore one/two-side/column
%
%    \begin{macrocode}
\DeclareOption{oneside}{\TBWarning{Option \CurrentOption\space ignored}}
\DeclareOption{twoside}{\ds@oneside}
\DeclareOption{onecolumn}{\ds@oneside}
\DeclareOption{twocolumn}{\ds@oneside}
%    \end{macrocode}
%
% There are these people who seem to think |tugproc| is an option rather
% than a class\dots{}  (Note that it's already been filtered out if we
% were calling from |ltugproc|.)
%
%    \begin{macrocode}
\DeclareOption{tugproc}{%
  \TBWarning{Option \CurrentOption\space ignored: use class ltugproc
    instead of \@tugclass}%
}
%    \end{macrocode}
%
% Option |rawcite| (the default) specifies the default citation
% mechanism (as built-in to \LaTeX); option |harvardcite| specifies
% the author-date citation mechanism defined in
% section~\ref{sec:citations} below.
%
%    \begin{macrocode}
\DeclareOption{rawcite}{\let\if@Harvardcite\iffalse}
\DeclareOption{harvardcite}{\let\if@Harvardcite\iftrue}
%    \end{macrocode}
%
% Option |extralabel| (the default) specifies that the publication
% years of two successive references with otherwise identical labels
% will be tagged with distinguishing letters; option |noextralabel|
% causes those letters to be suppressed.  Note that (a) no two
% references will in any case have the same labels in the default
% (plain) |rawcite| setup, and that (b) the distinguishing letters
% appear in the labels themselves~--- the even remotely intelligent
% reader should be able to work out the correspondence one with the
% other\dots
%
%    \begin{macrocode}
\DeclareOption{extralabel}{\let\UseExtraLabel\@firstofone}
\DeclareOption{noextralabel}{\let\UseExtraLabel\@gobble}
%    \end{macrocode}
%
% The section-numbering style, so that we can allow the same heading
% layout as in the plain macros.
% \changes{v1.4a}{1996/09/14}{Added nonumber and numbersec (default)
% options}
%
%    \begin{macrocode}
\DeclareOption{numbersec}{\let\if@numbersec\iftrue}
\DeclareOption{nonumber}{\let\if@numbersec\iffalse}
%    \end{macrocode}
% 
% Minimal running headers/footers contain just the TUGboat volume/issue
% identification and page numbers.  `runningfull' is the default, and
% includes title and author.  `runningoff' makes both headers and
% footers empty.
%
%    \begin{macrocode}
\DeclareOption{runningoff}{\AtEndOfClass{\@tubrunningoff}}
\DeclareOption{runningminimal}{\AtEndOfClass{\@tubrunningminimal}}
\DeclareOption{runningfull}{\AtEndOfClass{\@tubrunningfull}}
%    \end{macrocode}
%
% Any other options, we pass on to |article.cls| before we load it:
%
%    \begin{macrocode}
\DeclareOption*{\PassOptionsToClass{\CurrentOption}{article}}
%    \end{macrocode}
%
% Request default options (draft mode, standard citation, double-sided
% printing, etc.), process all options, and then get the base document
% class on top of which we reside.
%
%    \begin{macrocode}
\ExecuteOptions{draft,extralabel,numbersec,rawcite,runningminimal}
\ProcessOptions
\LoadClass[twoside]{article}
%    \end{macrocode}
%
% Various fonts used throughout.  Some effort has been made to
% suppress these things with explicit sizes in the macro name
% (\cs{tensl} is an example below), but keeping in step with the
% documentation is one thing that restricts such a move.
%
%    \begin{macrocode}
\def\sectitlefont{\fontfamily\sfdefault\fontseries{bx}\fontshape{n}%
         \fontsize\@xviipt\stbaselineskip\selectfont}
\def\tensl{\fontseries{m}\fontshape{sl}\fontsize\@xpt\@xiipt
          \selectfont}
%    \end{macrocode}
%
% This font selection command is used \emph{only} for the `Editor's
% Note' introduction to notes; sadly it makes explicit reference to
% \textsc{cmr}, and Barbara Beeton has agreed that the reference may be
% constructed to use the current family such that, if no upright
% italic is defined, ordinary italics are used.  A project for
% later\dots
%
%    \begin{macrocode}
\def\EdNoteFont{\fontfamily{cmr}\fontseries{m}\fontshape{ui}%
          \selectfont}
%</ltugboatcls>
%    \end{macrocode}
%
%
% If Ulrik Vieth's |mflogo.sty| is around, we'll use it.  Otherwise
% (pro tem, at least) we'll warn the user and define the absolute
% minimum of machinery that \TUB{} requires (that which was used
% prior to the invention of \LaTeXe).
%    \begin{macrocode}
%<*common>
\IfFileExists{mflogo.sty}%
  {\RequirePackage{mflogo}}%
%<!ltugcomn> {\TBWarning
%<ltugcomn>  {\PackageWarning{ltugcomn}
     {Package mflogo.sty not available --\MessageBreak
       Proceeding to emulate mflogo.sty}
   \DeclareRobustCommand\logofamily{%
     \not@math@alphabet\logofamily\relax
     \fontencoding{U}\fontfamily{logo}\selectfont}
   \DeclareTextFontCommand{\textlogo}{\logofamily}
   \def\MF{\textlogo{META}\-\textlogo{FONT}\@}
   \def\MP{\textlogo{META}\-\textlogo{POST}\@}
   \DeclareFontFamily{U}{logo}{}
   \DeclareFontShape{U}{logo}{m}{n}{%
     <8><9>gen*logo%
     <10><10.95><12><14.4><17.28><20.74><24.88>logo10%
   }{}
   \DeclareFontShape{U}{logo}{m}{sl}{%
     <8><9>gen*logosl%
     <10><10.95><12><14.4><17.28><20.74><24.88>logosl10%
   }{}
   \DeclareFontShape{U}{logo}{m}{it}{%
     <->ssub*logo/m/sl%
   }{}%
  }
%    \end{macrocode}
%
% \subsection{Resetting at start of paper}
%
% \begin{macro}{\ResetCommands}
% \begin{macro}{\AddToResetCommands}
% \begin{macro}{\StartNewPaper}
% We store a set of commands that should be executed at the start of
% each paper, before any paper-specific customisation.  These commands
% (stored in the token register \cs{ResetCommands}) include things
% suc as resetting section and footnote numbers, re-establishing
% default settings of typesetting parameters, and so on.  The user (or
% more typically, editor) may execute the commands by using the
% command \cs{StartNewPaper}.  Things I've not yet thought of may be
% added to the list of commands, by
%    \begin{macrocode}
\newtoks\ResetCommands
\ResetCommands{%
  \setcounter{part}{0}%
  \setcounter{section}{0}%
  \setcounter{footnote}{0}%
  \authornumber\z@
}
\newcommand{\AddToResetCommands}[1]{%
  \AddToResetCommands\expandafter{\AddToResetCommands#1}%
}
%    \end{macrocode}
% \end{macro}
% \end{macro}
% \end{macro}
%
% \subsection{Helpful shorthand (common code with Plain styles)}
%
% \cs{makeescape}, \dots, \cs{makecomment} allow users to change the
% category code of a single character a little more easily.  These
% require that the character be addressed as a control sequence:
% e.g., |\makeescape\/| will make `|/|' an escape character.
%
%    \begin{macrocode}
%<*!latex>
\def\makeescape#1{\catcode`#1=0 }
\def\makebgroup#1{\catcode`#1=1 }
\def\makeegroup#1{\catcode`#1=2 }
\def\makemath  #1{\catcode`#1=3 }
%</!latex>
%<*latex>
\def\makeescape#1{\catcode`#1=\z@}
\def\makebgroup#1{\catcode`#1=\@ne}
\def\makeegroup#1{\catcode`#1=\tw@}
\def\makemath  #1{\catcode`#1=\thr@@}
%</latex>
\def\makealign #1{\catcode`#1=4 }
\def\makeeol   #1{\catcode`#1=5 }
\def\makeparm  #1{\catcode`#1=6 }
\def\makesup   #1{\catcode`#1=7 }
\def\makesub   #1{\catcode`#1=8 }
\def\makeignore#1{\catcode`#1=9 }
\def\makespace #1{\catcode`#1=10 }
\def\makeletter#1{\catcode`#1=11 }
\chardef\other=12
\let\makeother\@makeother
\def\makeactive#1{\catcode`#1=13 }
\def\makecomment#1{\catcode`#1=14 }
%    \end{macrocode}
% |\savecat#1| and |\restorecat#1| will save and restore the category
% of a given character.  These are useful in cases where one doesn't
% wish to localize the settings and therefore be required to globally
% define or set things.
%    \begin{macrocode}
\def\savecat#1{%
  \expandafter\xdef\csname\string#1savedcat\endcsname{\the\catcode`#1}}
\def\restorecat#1{\catcode`#1=\csname\string#1savedcat\endcsname}
%<!latex>\savecat\@
%<!latex>\makeletter\@
%    \end{macrocode}
%
% |\SaveCS#1| and |\RestoreCS#1| save and restore `meanings' of control
% sequences.  Again this is useful in cases where one doesn't want to
% localize or where global definitions clobber a control sequence which
% is needed later with its `old' definition.
%
%    \begin{macrocode}
\def\SaveCS#1{\expandafter\let\csname saved@@#1\expandafter\endcsname
  \csname#1\endcsname}
\def\RestoreCS#1{\expandafter\let\csname#1\expandafter\endcsname
  \csname saved@@#1\endcsname}
%    \end{macrocode}
% To distinguish between macro files loaded
%    \begin{macrocode}
\def\plaintubstyle{plain}
\def\latextubstyle{latex}
%    \end{macrocode}
%
% Control sequences that were first defined in \LaTeXe{} of
% 1995/06/01 (or later), but which we merrily use.  Only define if
% necessary:
%
%    \begin{macrocode}
\providecommand\hb@xt@{\hbox to}
\providecommand\textsuperscript[1]{\ensuremath{\m@th
                                     ^{\mbox{\fontsize\sf@size\z@
                                         \selectfont #1}}}}
%    \end{macrocode}
% (Note that that definition of \cs{textsuperscript} isn't robust, but
% probably doesn't need to be\dots  What's more, it doesn't appear in
% the mythical 2.09 version of the package.)
%
% \subsection{Abbreviations and logos}
% \label{abbr-logo}
%
% Font used for the METAFONT logo, etc.
%    \begin{macrocode}
\def\AllTeX{(\La\kern-.075em)\kern-.075em\TeX}
\def\AMS{American Mathematical Society}
\def\AmS{$\mathcal{A}$\kern-.1667em\lower.5ex\hbox
    {$\mathcal{M}$}\kern-.125em$\mathcal{S}$}
\def\AmSLaTeX{\AmS-\LaTeX}
\def\AmSTeX{\AmS-\TeX}
\def\ANSI{\acro{ANSI}}
\def\API{\acro{API}}
\def\ASCII{\acro{ASCII}}
\def\aw{\acro{A\kern.04em\raise.115ex\hbox{-}W}}
\def\AW{Addison\kern.1em-\penalty\z@\hskip\z@skip Wesley}
%
% make \BibTeX work in slanted contexts too; it's common in titles, and
% especially burdensome to hack in .bib files.
\def\Bib{%
  \ifdim \fontdimen1\font>0pt
     B{\SMC\SMC IB}%
  \else
     \textsc{Bib}%
  \fi
}
\def\BibTeX{\Bib\kern-.08em \TeX}
%
\def\BSD{\acro{BSD}}
\def\CandT{\textsl{Computers \& Typesetting}}
\def\CD{\acro{CD}}
\def\CJK{\acro{CJK}}
%    \end{macrocode}
% We place our \cs{kern} after \cs{-} so that it disappears if the
% hyphenation is taken:
%    \begin{macrocode}
\def\ConTeXt{C\kern-.0333emon\-\kern-.0667em\TeX\kern-.0333emt}
\def\CMkIV{\ConTeXt\ \MkIV}
\def\Cplusplus{C\plusplus}
\def\plusplus{\raisebox{.7ex}{$_{++}$}}
\def\CSS{\acro{CSS}}
\def\CSV{\acro{CSV}}
\def\CTAN{\acro{CTAN}}
\def\DTD{\acro{DTD}}
\def\DTK{\acro{DTK}}
\def\DVD{\acro{DVD}}
\def\DVI{\acro{DVI}}
\def\DVIPDFMx{\acro{DVIPDFM}$x$}
\def\DVItoVDU{DVIto\kern-.12em VDU}
\def\ECMA{\acro{ECMA}}
\def\EPS{\acro{EPS}}
\DeclareRobustCommand\eTeX{\ensuremath{\varepsilon}-\kern-.125em\TeX}
\DeclareRobustCommand\ExTeX{%
  \ensuremath{\textstyle\varepsilon_{\kern-0.15em\cal{X}}}\kern-.2em\TeX}
\def\FAQ{\acro{FAQ}}
\def\FTP{\acro{FTP}}
\def\Ghostscript{Ghost\-script}
\def\GNU{\acro{GNU}}
\def\GUI{\acro{GUI}}
\def\Hawaii{Hawai`i}
\def\HTML{\acro{HTML}}
\def\HTTP{\acro{HTTP}}
\def\IDE{\acro{IDE}}
\def\IEEE{\acro{IEEE}}
\def\ISBN{\acro{ISBN}}
\def\ISO{\acro{ISO}}
\def\ISSN{\acro{ISSN}}
\def\JPEG{\acro{JPEG}}
\def\JTeX{\leavevmode\hbox{\lower.5ex\hbox{J}\kern-.18em\TeX}}
\def\JoT{\textsl{The Joy of \TeX}}
\def\LAMSTeX{L\raise.42ex\hbox{\kern-.3em
                   $\m@th$\fontsize\sf@size\z@\selectfont
                   $\m@th\mathcal{A}$}%
    \kern-.2em\lower.376ex\hbox{$\m@th\mathcal{M}$}\kern-.125em
    {$\m@th\mathcal{S}$}-\TeX}
% This code
% is hacked from its definition of \cs{LaTeX}; it allows slants (for
% example) to propagate into the raised (small) `A':
%    \begin{macrocode}
\newcommand{\La}%
   {L\kern-.36em
        {\setbox0\hbox{T}%
         \vbox to\ht0{\hbox{$\m@th$%
                            \csname S@\f@size\endcsname
                            \fontsize\sf@size\z@
                            \math@fontsfalse\selectfont
                            A}%
                      \vss}%
        }}
%    \end{macrocode}
%
% We started with the intention that we wouldn't redefine \cs{LaTeX}
% when we're running under it, so as not to trample on an existing
% definition.  However, this proves less than satisfactory; a single
% logo may be OK for the run of documents, but for \TUB{}, we find
% that something noticeably better is necessary; see section
% \ref{sec:latex-logo}.
%
%    \begin{macrocode}
%<!latex>\def\LaTeX{\La\kern-.15em\TeX}
\def\LyX{L\kern-.1667em\lower.25em\hbox{Y}\kern-.125emX}
\def\MacOSX{Mac\,\acro{OS\,X}}
\def\MathML{Math\acro{ML}}
\def\Mc{\setbox\TestBox=\hbox{M}M\vbox
   to\ht\TestBox{\hbox{c}\vfil}} %  for Robert McGaffey
%    \end{macrocode}
%
% If we're running under \LaTeXe, we're using (at least pro tem) Ulrik
% Vieth's |mflogo.sty| if it's present.  Otherwise, we're using a
% short extract of Vieth's stuff.  Either way, we don't need to
% specify \cs{MF} or \cs{MP}
%    \begin{macrocode}
\def\mf{\textsc{Metafont}}
\def\MFB{\textsl{The \MF book}}
\def\MkIV{Mk\acro{IV}}
\let\TB@@mp\mp
\DeclareRobustCommand\mp{\ifmmode\TB@@mp\else MetaPost\fi}
%
% In order that the \cs{OMEGA} command will switch to using the TS1
% variant of the capital Omega character if \texttt{textcomp.sty} is
% loaded, we define it in terms of the \cs{textohm} command.  Note
% that this requires us to interpose a level of indirection, rather
% than to use \cs{let}\dots
%
%    \begin{macrocode}
\DeclareRobustCommand{\NTG}{\acro{NTG}}
\DeclareRobustCommand\NTS{\ensuremath{\mathcal{N}\mkern-4mu
  \raisebox{-0.5ex}{$\mathcal{T}$}\mkern-2mu \mathcal{S}}}
\DeclareTextSymbol{\textohm}{OT1}{'012}
\DeclareTextSymbolDefault{\textohm}{OT1}
\newcommand\OMEGA{\textohm}
\DeclareRobustCommand{\OCP}{\OMEGA\acro{CP}}
\DeclareRobustCommand{\OOXML}{\acro{OOXML}}
\DeclareRobustCommand{\OTF}{\acro{OTF}}
\DeclareRobustCommand{\OTP}{\OMEGA\acro{TP}}
\def\mtex{T\kern-.1667em\lower.424ex\hbox{\^E}\kern-.125emX\@}
%    \end{macrocode}
% Revised definition of \cs{NTS} based on that used by Phil Taylor.
%    \begin{macrocode}
\def\Pas{Pascal}
\def\pcMF{\leavevmode\raise.5ex\hbox{p\kern-.3\p@ c}MF\@}
\def\PCTeX{PC\thinspace\TeX}
\def\pcTeX{\leavevmode\raise.5ex\hbox{p\kern-.3\p@ c}\TeX}
\def\PDF{\acro{PDF}}
\def\PGF{\acro{PGF}}
\def\PHP{\acro{PHP}}
\def\PiC{P\kern-.12em\lower.5ex\hbox{I}\kern-.075emC\@}
\def\PiCTeX{\PiC\kern-.11em\TeX}
\def\plain{\texttt{plain}}
\def\PNG{\acro{PNG}}
\def\POBox{P.\thinspace O.~Box }
\def\PS{{Post\-Script}}
\def\PSTricks{\acro{PST}ricks}
\def\RTF{\acro{RTF}}
\def\SC{Steering Committee}
\def\SGML{\acro{SGML}}
\def\SliTeX{\textrm{S\kern-.06em\textsc{l\kern-.035emi}%
                     \kern-.06em\TeX}}
\def\slMF{\textsl{\MF}} % should never be used
\def\SQL{\acro{SQL}}
\def\stTeX{\textsc{st}\kern-0.13em\TeX}
\def\STIX{\acro{STIX}}
\def\SVG{\acro{SVG}}
\def\TANGLE{\texttt{TANGLE}\@}
\def\TB{\textsl{The \TeX book}}
\def\TIFF{\acro{TIFF}}
\def\TP{\textsl{\TeX}: \textsl{The Program}}
\DeclareRobustCommand\TeX{T\kern-.1667em\lower.424ex\hbox{E}\kern-.125emX\@}
\def\TeXhax{\TeX hax}
\def\TeXMaG{\TeX M\kern-.1667em\lower.5ex\hbox{A}%
   \kern-.2267emG\@}
\def\TeXtures{\textit{Textures}}
\let\Textures=\TeXtures
\def\TeXworks{\TeX\kern-.07em works}
\def\TeXXeT{\TeX-{}-\XeT}
\def\TFM{\acro{TFM}}
\expandafter\ifx\csname XeTeXrevision\endcsname\relax
\def\Thanh{H\`an~Th\^e\llap{\raise 0.5ex\hbox{\'{}}}~Th\`anh}% non-XeTeX
\else
\def\Thanh{H\`an~Th\textcircumacute{e}~Th\`anh}% xunicode drops the acute else
\fi
\def\TikZ{Ti{\em k}Z}
\def\ttn{\textsl{TTN}\@}
\def\TTN{\textsl{\TeX{} and TUG News}}
\let\texttub\textsl              % redefined in other situations
\def\TUB{\texttub{TUGboat}}
\def\TUG{\TeX\ \UG}
\def\tug{\acro{TUG}}
\def\UG{Users Group}
\def\UNIX{\acro{UNIX}}
\def\UTF{\acro{UTF}}
\def\VAX{V\kern-.12em A\kern-.1em X\@}
\def\VnTeX{V\kern-.03em n\kern-.02em \TeX}
\def\VorTeX{V\kern-2.7\p@\lower.5ex\hbox{O\kern-1.4\p@ R}\kern-2.6\p@\TeX}
\def\XeT{X\kern-.125em\lower.424ex\hbox{E}\kern-.1667emT\@}
\def\XML{\acro{XML}}
\def\WEB{\texttt{WEB}\@}
\def\WEAVE{\texttt{WEAVE}\@}
\def\WYSIWYG{\acro{WYSIWYG}}
%    \end{macrocode}
%
% Xe\TeX{} requires reflecting the first E, hence we complain if the
% graphics package is not present.  (For plain documents, this can be
% loaded via Eplain.)  Also, at Barbara's suggestion, if the current
% font is slanted, we rotate by 180 instead of reflecting so there is at
% least a chance to look ok.  (The magic values here seem more or less
% ok for \texttt{cmsl} and \texttt{cmti}.)
%
%    \begin{macrocode}
\def\tubreflect#1{%
  \@ifundefined{reflectbox}{%
    \TBerror{A graphics package must be loaded for \string\XeTeX}%
  }{%
    \ifdim \fontdimen1\font>0pt
      \raise 1.75ex \hbox{\kern.1em\rotatebox{180}{#1}}\kern-.1em
    \else
      \reflectbox{#1}%
    \fi
  }%
}
\def\tubhideheight#1{\setbox0=\hbox{#1}\ht0=0pt \dp0=0pt \box0 }
\def\XekernbeforeE{-.125em}
\def\XekernafterE{-.1667em}
\DeclareRobustCommand\Xe{\leavevmode
  \tubhideheight{\hbox{X%
    \setbox0=\hbox{\TeX}\setbox1=\hbox{E}%
    \lower\dp0\hbox{\raise\dp1\hbox{\kern\XekernbeforeE\tubreflect{E}}}%
    \kern\XekernafterE}}}
\def\XeTeX{\Xe\TeX}
\def\XeLaTeX{\Xe{\kern.11em \LaTeX}}
%
\def\XHTML{\acro{XHTML}}
\def\XSL{\acro{XSL}}
\def\XSLFO{\acro{XSL}\raise.08ex\hbox{-}\acro{FO}}
\def\XSLT{\acro{XSLT}}
%    \end{macrocode}
%
% \subsection{General typesetting rules}
%
%    \begin{macrocode}
\newlinechar=`\^^J
\normallineskiplimit=\p@
\clubpenalty=10000
\widowpenalty=10000
\def\NoParIndent{\parindent=\z@}
\newdimen\normalparindent
\normalparindent=20\p@
\def\NormalParIndent{\global\parindent=\normalparindent}
\NormalParIndent
\def\BlackBoxes{\overfullrule=5\p@}
\def\NoBlackBoxes{\overfullrule=\z@}
\def\newline{\hskip\z@\@plus\pagewd\break}
%    \end{macrocode}
%
% Hyphen control: first, we save the hyphenpenalties in
% \cs{allowhyphens}.  This allows us to permit hyphens temporarily in
% things like \cs{netaddress}es, which typically occur when
% \cs{raggedright} is set,  but which need to be allowed to break at
% their artificial discretionaries.
%
%    \begin{macrocode}
\edef\allowhyphens{\noexpand\hyphenpenalty\the\hyphenpenalty\relax
  \noexpand\exhyphenpenalty\the\exhyphenpenalty\relax}
\def\nohyphens{\hyphenpenalty\@M\exhyphenpenalty\@M}
%    \end{macrocode}
%
% \subsection{Utility registers and definitions}
%
% We define a few scratch registers (and the like) for transient use;
% they're all paired: an internal one (\cs{T@st*}) and an external one
% (\cs{Test*}).
%
% \begin{description}
% \item[\emph{Comment:}] Exercise for an idle day: find whether all
%   these are necessary, or whether we can use the \LaTeX{} temporaries
%   for some (or all) of the \cs{T@st*} ones.
% \item[\emph{Comment:}] (bb) All these registers are used in the
%   plain version, |tugboat.sty|.
% \end{description}
%
%    \begin{macrocode}
\newbox\T@stBox                 \newbox\TestBox
\newcount\T@stCount             \newcount\TestCount
\newdimen\T@stDimen             \newdimen\TestDimen
\newif\ifT@stIf                 \newif\ifTestIf
%    \end{macrocode}
%
% Control sequence existence test, stolen from \TeX book exercise 7.7
% (note that this provides functionality that in some sense duplicates
% something within \LaTeX).
%
%    \begin{macrocode}
\def\ifundefined#1{\expandafter\ifx\csname#1\endcsname\relax }
%    \end{macrocode}
%
% \LaTeX\ conventions which are also useful here.
%
%    \begin{macrocode}
%<*!latex>
  \let\@@input\input
  \def\iinput#1{\@@input#1 }
  \def\@inputcheck{\if\@nextchar\bgroup
    \expandafter\iinput\else\expandafter\@@input\fi}
  \def\input{\futurelet\@nextchar\@inputcheck}
%</!latex>
%    \end{macrocode}
%
% Smashes repeated from AMS-\TeX; plain \TeX{} implements only full
% |\smash|.
%
%    \begin{macrocode}
\newif\iftop@           \newif\ifbot@
\def\topsmash{\top@true\bot@false\smash@}
\def\botsmash{\top@false\bot@true\smash@}
\def\smash{\top@true\bot@true\smash@}
\def\smash@{\relax\ifmmode\def\next{\mathpalette\mathsm@sh}%
        \else\let\next\makesm@sh\fi \next }
\def\finsm@sh{\iftop@\ht\z@\z@\fi\ifbot@\dp\z@\z@\fi\box\z@}
%    \end{macrocode}
%
%       Vertical `laps'; cf.\ |\llap| and |\rlap|
%
%    \begin{macrocode}
\long\def\ulap#1{\vbox to \z@{\vss#1}}
\long\def\dlap#1{\vbox to \z@{#1\vss}}
%    \end{macrocode}
%       And centered horizontal and vertical `laps'
%    \begin{macrocode}
\def\xlap#1{\hb@xt@\z@{\hss#1\hss}}
\long\def\ylap#1{\vbox to \z@{\vss#1\vss}}
\long\def\zlap#1{\ylap{\xlap{#1}}}
%    \end{macrocode}
%       Avoid unwanted vertical glue when making up pages.
%    \begin{macrocode}
\def\basezero{\baselineskip\z@skip \lineskip\z@skip}
%    \end{macrocode}
%  Empty rules for special occasions
%    \begin{macrocode}
\def\nullhrule{\hrule \@height\z@ \@depth\z@ \@width\z@ }
\def\nullvrule{\vrule \@height\z@ \@depth\z@ \@width\z@ }
%    \end{macrocode}
%       Support ad-hoc strut construction.
%    \begin{macrocode}
\def\makestrut[#1;#2]{\vrule \@height#1 \@depth#2 \@width\z@ }
%    \end{macrocode}
%       Construct box for figure pasteup, etc.;
%       height = \#1, width = \#2, rule thickness = \#3
%    \begin{macrocode}
\def\drawoutlinebox[#1;#2;#3]{\T@stDimen=#3
        \vbox to#1{\hrule \@height\T@stDimen \@depth\z@
            \vss\hb@xt@#2{\vrule \@width\T@stDimen
                \hfil\makestrut[#1;\z@]%
                \vrule \@width\T@stDimen}\vss
            \hrule \@height\T@stDimen \@depth\z@}}
%    \end{macrocode}
%       Today's date, to be printed on drafts.  Based on \TeX book, p.406.
%    \begin{macrocode}
%<*!latex>
\def\today{\number\day\space \ifcase\month\or
        Jan \or Feb \or Mar \or Apr \or May \or Jun \or
        Jul \or Aug \or Sep \or Oct \or Nov \or Dec \fi
        \number\year}
%</!latex>
%    \end{macrocode}
%       Current time; this may be system dependent!
%    \begin{macrocode}
\newcount\hours
\newcount\minutes
\def\SetTime{\hours=\time
        \global\divide\hours by 60
        \minutes=\hours
        \multiply\minutes by 60
        \advance\minutes by-\time
        \global\multiply\minutes by-1 }
\SetTime
\def\now{\number\hours:\ifnum\minutes<10 0\fi\number\minutes}
\def\Now{\today\ \now}
\newif\ifPrelimDraft
\def\midrtitle{\ifPrelimDraft {\textsl{preliminary draft, \Now}}\fi}
%    \end{macrocode}
%
% \subsection{Ragged right and friends}
%
% \begin{macro}{\raggedskip}
% \begin{macro}{\raggedstretch}
% \begin{macro}{\raggedparfill}
% \begin{macro}{\raggedspaces}
% Plain \TeX{}'s definition of |\raggedright| doesn't permit any
% stretch, and results in too many overfull boxes.  We also turn off
% hyphenation.  This code lies somewhere between that of Plain \TeX{}
% and of \LaTeX{}.
%    \begin{macrocode}
\newdimen\raggedskip    \raggedskip=\z@
\newdimen\raggedstretch \raggedstretch=5em    % ems of font set now (10pt)
\newskip\raggedparfill  \raggedparfill=\z@\@plus 1fil
\def\raggedspaces{\spaceskip=.3333em \relax \xspaceskip=.5em \relax }
%    \end{macrocode}
% \end{macro}
% \end{macro}
% \end{macro}
% \end{macro}
%
% \begin{macro}{\raggedright}
% \begin{macro}{\raggedleft}
% \begin{macro}{\raggedcenter}
% \begin{macro}{\normalspaces}
% Some applications may have to add stretch, in order to avoid
% all overfull boxes.  We define the following uses of the above
% skips, etc.
%    \begin{macrocode}
\def\raggedright{%
  \nohyphens
  \rightskip=\raggedskip\@plus\raggedstretch \raggedspaces
  \parfillskip=\raggedparfill
}
\def\raggedleft{%
  \nohyphens
  \leftskip=\raggedskip\@plus\raggedstretch \raggedspaces
  \parfillskip=\z@skip
}
\def\raggedcenter{%
  \nohyphens
  \leftskip=\raggedskip\@plus\raggedstretch
  \rightskip=\leftskip \raggedspaces
  \parindent=\z@ \parfillskip=\z@skip
}
\def\normalspaces{\spaceskip\z@skip \xspaceskip\z@skip}
%    \end{macrocode}
% \end{macro}
% \end{macro}
% \end{macro}
% \end{macro}
%
% Miscellaneous useful stuff.  Note that \LaTeXe{} defines a robust
% \cs{,}, but that we provide a new definition of |~| by redefining
% its robust underpinnings\footnote{\cs{DeclareRobustCommand} doesn't
%   mind redefinition, fortunately} (based on the version in
% AMS-\TeX~--- the \LaTeXe{} version has \cs{leavevmode} and doesn't
% care about surrounding space).
%    \begin{macrocode}
\DeclareRobustCommand{\nobreakspace}{%
  \unskip\nobreak\ \ignorespaces}
%    \end{macrocode}
%
% Plain \TeX{} defines \cs{newbox} as \cs{outer}.  We solemnly
% preserve the following, which removes the \cs{outer}ness; of course,
% we carefully exclude it from what we generate\dots{}  (\cs{outer}ness is a
% spawn of the devil, is it not?  Barbara Beeton responded to the
% previous sentence ``\cs{outer}ness has its place: it avoids register
% buildup, hence running out of memory''.  In another context, David
% Carlisle remarked that an error control mechanism that causes more
% confusing errors than it prevents is rather a poor one.  This is
% perhaps not the place to conduct a serious debate\dots)
%
%    \begin{macrocode}
\def\boxcs#1{\box\csname#1\endcsname}
\def\setboxcs#1{\setbox\csname#1\endcsname}
\def\newboxcs#1{\expandafter\newbox\csname#1\endcsname}
\let\gobble\@gobble
\def\vellipsis{%
  \leavevmode\kern0.5em
  \raise\p@\vbox{\baselineskip6\p@\vskip7\p@\hbox{.}\hbox{.}\hbox{.}}
  }
\def\bull{\vrule \@height 1ex \@width .8ex \@depth -.2ex }
\def\cents{{\rm\raise.2ex\rlap{\kern.05em$\scriptstyle/$}c}}
\def\careof{\leavevmode\hbox{\raise.75ex\hbox{c}\kern-.15em
                /\kern-.125em\smash{\lower.3ex\hbox{o}}} \ignorespaces}
\def\Dag{\raise .6ex\hbox{$\scriptstyle\dagger$}}
%
\DeclareRobustCommand\sfrac[1]{\@ifnextchar/{\@sfrac{#1}}%
                                            {\@sfrac{#1}/}}
\def\@sfrac#1/#2{\leavevmode\kern.1em\raise.5ex
         \hbox{$\m@th\mbox{\fontsize\sf@size\z@
                           \selectfont#1}$}\kern-.1em
         /\kern-.15em\lower.25ex
          \hbox{$\m@th\mbox{\fontsize\sf@size\z@
                            \selectfont#2}$}}
%
% don't stay bold in description items, bold italic is too weird.
\DeclareRobustCommand\meta[1]{%
  \ensuremath{\langle}%
  \ifmmode \mbox\bgroup \fi % if in math
  {\it #1\/}% no typewriter italics, please
  \ifmmode \egroup \fi
  \ensuremath{\rangle}%
}
%
\DeclareRobustCommand\cs[1]{\texttt{\char`\\#1}}
%
\DeclareRobustCommand\env[1]{%
  \cs{begin}\texttt{\char`\{#1\char`\}}}
%
\def\thinskip{\hskip 0.16667em\relax}
%    \end{macrocode}
%
% We play a merry game with dashes, providing all conceivable options
% of breakability before and after.
%
%    \begin{macrocode}
\def\endash{--}
\def\emdash{\endash-}
\def\d@sh#1#2{\unskip#1\thinskip#2\thinskip\ignorespaces}
\def\dash{\d@sh\nobreak\endash}
\def\Dash{\d@sh\nobreak\emdash}
\def\ldash{\d@sh\empty{\hbox{\endash}\nobreak}}
\def\rdash{\d@sh\nobreak\endash}
\def\Ldash{\d@sh\empty{\hbox{\emdash}\nobreak}}
\def\Rdash{\d@sh\nobreak\emdash}
%    \end{macrocode}
%
% Hacks to permit automatic hyphenation after an actual hyphen, or
% after a slash.
%
%    \begin{macrocode}
\def\hyph{-\penalty\z@\hskip\z@skip }
\def\slash{/\penalty\z@\hskip\z@skip }
%    \end{macrocode}
%
% Adapted from |comp.text.tex| posting by Donald Arseneau, 26 May 93.
% \LaTeXe-isation added by Robin Fairbairns.  Destroys both the
% |TestCount|s.
%
%    \begin{macrocode}
\def\nth#1{%
    \def\reserved@a##1##2\@nil{\ifcat##1n%
           0%
          \let\reserved@b\ensuremath
      \else##1##2%
          \let\reserved@b\relax
      \fi}%
    \TestCount=\reserved@a#1\@nil\relax
    \ifnum\TestCount <0 \multiply\TestCount by\m@ne \fi % subdue negatives
    \T@stCount=\TestCount
    \divide\T@stCount by 100 \multiply\T@stCount by 100
    \advance\TestCount by-\T@stCount     % n mod 100
    \ifnum\TestCount >20 \T@stCount=\TestCount
      \divide\T@stCount by 10 \multiply\T@stCount by 10
      \advance\TestCount by-\T@stCount   % n mod 10
    \fi
     \reserved@b{#1}%
       \textsuperscript{\ifcase\TestCount th%    0th
                        \or   st%                1st
                        \or   nd%                2nd
                        \or   rd%                3rd
                        \else th%                nth
                        \fi}%
}
%    \end{macrocode}
%
% \subsection{Reviews}
%
% Format information on reviewed items for book review articles.
% For the \LaTeXe{} version, we follow Fairbairns' maxim, and define
% something that can even look like a \LaTeX{} macro\dots
%
%    \begin{macrocode}
\def\Review{\@ifnextchar:{\@Review}{\@Review:}}
\def\@Review:{\@ifnextchar[%]
  {\@Rev}%
  {\@Rev[Book review]}}
\def\@Rev[#1]#2{{\ignorespaces#1\unskip:\enspace\ignorespaces
                                        \slshape\mdseries#2}}
\def\reviewitem{\addvspace{\BelowTitleSkip}%
  \def\revauth##1{\def\therevauth{##1, }\ignorespaces}%
  \def\revtitle##1{\def\therevtitle{{\slshape##1}. }\ignorespaces}%
  \def\revpubinfo##1{\def\therevpubinfo{##1.}\ignorespaces}%
}
\def\endreviewitem{{\noindent\interlinepenalty=10000
  \therevauth\therevtitle\therevpubinfo\endgraf}%
  \vskip\medskipamount
}
\def\booktitle#1{{\slshape#1\/}}
%    \end{macrocode}
%
%  \subsection{Dates, volume and issue numbers, etc.}
%
% Dates and other items which identify the volume and issue.
% \cs{issueseqno} is a sequential issue number starting from the
% first issue published; volume 15,4 has \cs{issueseqno=45}.
%
% To use: \begin{tabular}{l}
%              |\vol 19, 1.|\\
%              |\issdate March 1998.|\\
%              |\issueseqno=58|\\
%         \end{tabular}
%
% Starting with volume 23 (nominal 2002), we have \cs{issyear} instead
% of \cs{issdate}, because issues don't have months any more.
%
% For production, these are set in a separate file,
% |tugboat.dates|, which is issue-specific.
%
% \begin{description}
% \item[\emph{Comment:}] I would like to make the code read a file
%   |tugboat.dates| in the current directory or its parent.
%   This is easy except under `odd' operating systems ({\small VMS} is
%   an example that springs to mind, {\small RISC}os may be even
%   worse) whose syntax is out of the ordinary.
% \end{description}
%
%    \begin{macrocode}
\newcount\issueseqno            \issueseqno=-1
\def\v@lx{\gdef\volx{Volume~\volno~(\volyr), No.~\issno}}
\def\volyr{}
\def\volno{}
\def\vol #1,#2.{\gdef\volno{#1\unskip}%
        \gdef\issno{\ignorespaces#2\unskip}%
        \setbox\TestBox=\hbox{\volyr}%
        \ifdim \wd\TestBox > .2em \v@lx \fi }
\def\issyear #1.{\gdef\issdt{#1}\gdef\volyr{#1}%
        \gdef\bigissdt{#1}%
        \setbox\TestBox=\hbox{\volno}%
        \ifdim \wd\TestBox > .2em \v@lx \fi }
\def\issdate #1#2 #3.{\gdef\issdt{#1#2 #3}\gdef\volyr{#3}%
        \gdef\bigissdt{#1{\smc\uppercase{#2}} #3}%
        \setbox\TestBox=\hbox{\volno}%
        \ifdim \wd\TestBox > .2em \v@lx \fi }
\vol 0, 0.
\issdate Thermidor, 9999.
%    \end{macrocode}
%
% (The curious should know that \emph{Thermidor} was one of the French
% revolutionary month names\dots)
%
% For \LaTeX{} use, define a version of the issue declaration that can
% take or leave the old plain syntax
%
%    \begin{macrocode}
%<!latex>\def\tubissue#1(#2)%
%<*latex>
\def\tubissue#1{\@ifnextchar(%)
  {\@tubissue@b{#1}}
  {\@tubissue@a{#1}}}
\def\@tubissue@b#1(#2){\@tubissue@a{#1}{#2}}
\def\@tubissue@a#1#2%
%</latex>
  {\TUB~#1, no.~#2}
%    \end{macrocode}
%
%  \TUB{} conventions include the sequential issue number in the file name.
%  Permit this to be incorporated into file names automatically.
%  If \(\mbox{issue number} = 11\), \cs{Input filnam} will read
%  |tb11filnam.tex|
%
%    \begin{macrocode}
\def\infil@{\jobname}
\def\Input #1 {\ifnum\issueseqno<0
    \def\infil@{#1}%
  \else
    \def\infil@{tb\number\issueseqno#1}
  \fi
  \edef\jobname{\infil@}\@readFLN
  \@@input \infil@\relax
  \if@RMKopen
    \immediate\closeout\@TBremarkfile\@RMKopenfalse
  \fi
}
%    \end{macrocode}
%
% \cs{TBremark}s are things that need to be drawn to the attention of
% the editors; the conscientious author will include such things in
% the article file.  By default, remarks are suppressed, but their
% appearance may be enabled by the \cs{TBEnableRemarks} command, which
% can be included in the configuration file |ltugboat.cfg| (or
% |ltugproc.cfg|, if that's what we're at).
%
%    \begin{macrocode}
\newif\if@RMKopen        \@RMKopenfalse
\newwrite\@TBremarkfile
\def\@TBremark#1{%
  \if@RMKopen
  \else
    \@RMKopentrue\immediate\openout\@TBremarkfile=\infil@.rmk
  \fi
  \toks@={#1}%
  \immediate\write\@TBremarkfile{^^J\the\toks@}%
  \immediate\write16{^^JTBremark:: \the\toks@^^J}%
}
%    \end{macrocode}
%
% We initialise \cs{TBremark} to ignore its argument (this used to
% involve a \cs{TBremarkOFF} which was cunningly defined exactly the
% same as \cs{gobble})
%
%    \begin{macrocode}
\let\TBremark=\gobble
%    \end{macrocode}
%
% \cs{TBEnableRemarks} simply involves setting \cs{TBremark} to use
% the functional \cs{@TBremark} defined above.
%
%    \begin{macrocode}
\def\TBEnableRemarks{\let\TBremark\@TBremark}
%    \end{macrocode}
%
% For marking locations in articles that pertain to remarks
% in another file of editorial comments
%
%    \begin{macrocode}
\def\TUBedit#1{}
%    \end{macrocode}
%
% For using different filenames in the production process than those
% supplied by authors
%
%    \begin{macrocode}
\def\TUBfilename#1#2{\expandafter\def\csname file@@#1\endcsname{#2}}
\newread\@altfilenames
\def\@readFLN{\immediate\openin\@altfilenames=\jobname.fln
  \ifeof\@altfilenames\let\@result\relax\else
  \def\@result{\@@input\jobname.fln }\fi
  \immediate\closein\@altfilenames
  \@result}
\@readFLN
\everyjob=\expandafter{\the\everyjob\@readFLN}
\InputIfFileExists{\jobname.fln}%
     {\TBInfo{Reading alternative file file \jobname.fln}}{}
%    \end{macrocode}
%
% The following needs to work entirely in \TeX's mouth
%    \begin{macrocode}
\def\@tubfilename#1{\expandafter\ifx\csname file@@#1\endcsname\relax
  #1\else\csname file@@#1\endcsname\fi}
\def\fileinput#1{\@@input\@tubfilename{#1} }
%    \end{macrocode}
%
% Write out (both to a file and to the log) the starting page number
% of an article, to be used for cross references and in contents.
% |\pagexref|  is used for articles fully processed in the \TUB{}
% run.  |\PageXref|  is used for `extra' pages, where an item is
% submitted as camera copy, and only running heads (at most) are run.
%
%    \begin{macrocode}
%<*!latex>
\def\pagexrefON#1{%
        \write-1{\def\expandafter\noexpand\csname#1\endcsname{\number\pageno}}%
        \write\ppoutfile{%
                \def\expandafter\noexpand\csname#1\endcsname{\number\pageno}}%
        }
\def\PageXrefON#1{%
        \immediate\write-1{\def\expandafter
                        \noexpand\csname#1\endcsname{\number\pageno}}%
        \immediate\write\ppoutfile{\def\expandafter
                        \noexpand\csname#1\endcsname{\number\pageno}}}
%</!latex>
%<*latex>
\def\pagexrefON#1{%
        \write-1{\def\expandafter\noexpand\csname#1\endcsname{\number\c@page}}%
        \write\ppoutfile{%
                \def\expandafter\noexpand\csname#1\endcsname{\number\c@page}}%
        }
\def\PageXrefON#1{%
        \immediate\write-1{\def\expandafter
                        \noexpand\csname#1\endcsname{\number\c@page}}%
        \immediate\write\ppoutfile{\def\expandafter
                        \noexpand\csname#1\endcsname{\number\c@page}}}
%</latex>
\def\pagexrefOFF#1{}
\let\pagexref=\pagexrefOFF
\def\PageXrefOFF#1{}
\let\PageXref=\PageXrefOFF
\def\xreftoON#1{%
  \ifundefined{#1}%
    ???\TBremark{Need cross reference for #1.}%
  \else\csname#1\endcsname\fi}
\def\xreftoOFF#1{???}
\let\xrefto=\xreftoOFF
%    \end{macrocode}
%
% \cs{TBdriver} `marks code for use when articles are run together in
% a driver file'.  Since we don't yet have a definition of that
% arrangement, we don't have a definition of \cs{TBdriver}.  Its
% argument (which one presumes was intended as the code for this
% unusual state) is just gobbled.
%
%    \begin{macrocode}
\let\TBdriver\gobble
%    \end{macrocode}
%
% Some hyphenation exceptions:
%
%    \begin{macrocode}
\ifx\tubomithyphenations\@thisisundefined
\hyphenation{Del-a-ware Dijk-stra Duane Eijk-hout
  Flor-i-da Free-BSD Ghost-script Ghost-view
  Hara-lam-bous Jac-kow-ski Karls-ruhe
  Mac-OS Ma-la-ya-lam Math-Sci-Net
  Net-BSD Open-BSD Open-Office
  Pfa-Edit Post-Script Rich-ard Skoup South-all
  Vieth VM-ware Win-Edt
  acro-nym acro-nyms analy-sis ap-pen-di-ces ap-pen-dix asyn-chro-nous
  bib-lio-graph-i-cal bit-map bit-mapped bit-maps buf-fer buf-fers bool-ean
  col-umns com-put-able com-put-abil-ity cus-tom-iz-able
  data-base data-bases
   de-allo-cate de-allo-cates de-allo-cated de-allo-ca-tion
   de-riv-a-tive de-riv-a-tives de-riv-a-ble der-i-va-tion dis-trib-ut-able
  es-sence
  fall-ing
  half-way
  in-fra-struc-ture
  key-note
  long-est
  ma-gyar man-u-script man-u-scripts meta-table meta-tables
   mne-mon-ic mne-mon-ics mono-space mono-spaced
  name-space name-spaces
  off-line over-view
  pal-ettes par-a-digm par-a-dig-mat-ic par-a-digms
   pipe-line pipe-lines
   plug-in plug-ins pres-ent-ly pro-gram-mable
  re-allo-cate re-allo-cates re-allo-cated re-printed
  set-ups se-vere-ly spell-ing spell-ings stand-alone strong-est
   sub-ex-pres-sion sub-tables sur-gery syn-chro-ni-city syn-chro-nous
  text-height text-length text-width
   time-stamp time-stamped time-stamps
  vis-ual vis-ual-ly
  which-ever white-space white-spaces wide-spread wrap-around
}
\fi
%<!latex>\restorecat\@
%</common>
%<*classtail>
\PrelimDrafttrue
%    \end{macrocode}
% \subsection{Page dimensions, glue, penalties etc}
%    \begin{macrocode}
\textheight 54pc
\textwidth 39pc
\columnsep 1.5pc
\columnwidth 18.75pc
\parindent \normalparindent
\parskip \z@ % \@plus\p@
\leftmargini 2em
\leftmarginv .5em
\leftmarginvi .5em
\oddsidemargin \z@
\evensidemargin \z@
\topmargin -2.5pc
\headheight 12\p@
\headsep 20\p@
\marginparwidth 48\p@
\marginparsep 10\p@
\partopsep=\z@
\topsep=3\p@\@plus\p@\@minus\p@
\parsep=3\p@\@plus\p@\@minus\p@
\itemsep=\parsep
\twocolumn
\newdimen\pagewd        \pagewd=39pc
\newdimen\trimwd        \trimwd=\pagewd
\newdimen\trimlgt       \trimlgt=11in
\newdimen\headmargin    \headmargin=3.5pc
%    \end{macrocode}
%
% In \LaTeXe, |twoside| option is forced on when
% |article.cls| is loaded.
%
% \subsection{Messing about with the \LaTeX{} logo}
% \label{sec:latex-logo}
%
% Barbara Beeton's pleas for \LaTeX{} logos that look right in any
% font shape provoked me to generate the following stuff that is
% configurable.
%
% Here's the command for the user to define hir own new version.  The
% arguments are font family, series and shape, and then the two kern
% values used in placing the raised `A' of \LaTeX.
%
%    \begin{macrocode}
\newcommand\DeclareLaTeXLogo[5]{\expandafter\def
  \csname @LaTeX@#1/#2/#3\endcsname{{#4}{#5}}}
%    \end{macrocode}
%
% The default values are as used in the source of \LaTeX{} itself:
%
%    \begin{macrocode}
\def\@LaTeX@default{{.36}{.15}}
%    \end{macrocode}
%
% More are defined in the initial version, for bold CM sans (which is
% used as \cs{SecTitleFont}), and CM italic medium and bold, and
% Bitstream Charter (which Nelson Beebe likes to use):
%
%    \begin{macrocode}
\DeclareLaTeXLogo{cmss}{bx}n{.3}{.15}
\DeclareLaTeXLogo{cmr}m{it}{.3}{.27}
\DeclareLaTeXLogo{cmr}{bx}{it}{.3}{.27}
\DeclareLaTeXLogo{bch}{m}{n}{.2}{.08}
\DeclareLaTeXLogo{bch}{m}{it}{.2}{.08}
%    \end{macrocode}
%
% Redefine \cs{LaTeX} to choose the parameters for the current font,
% or to use the default value otherwise:
%
%    \begin{macrocode}
\DeclareRobustCommand\LaTeX{\expandafter\let\expandafter\reserved@a
  \csname @LaTeX@\f@family/\f@series/\f@shape\endcsname
  \ifx\reserved@a\relax\let\reserved@a\@LaTeX@default\fi
  \expandafter\@LaTeX\reserved@a}
%    \end{macrocode}
%
% Here's the body of what was originally \cs{LaTeX}, pulled out with its
% roots dripping onto the smoking ruin of original \LaTeX, and then
% bits stuck in on the side.
%
% \cs{@LaTeX@default} provides parameters as one finds in the
% original; other versions are added as needed.
%
%    \begin{macrocode}
\newcommand\@LaTeX[2]{L\kern-#1em
        {\sbox\z@ T%
         \vbox to\ht0{\hbox{$\m@th$%
                            \csname S@\f@size\endcsname
                            \fontsize\sf@size\z@
                            \math@fontsfalse\selectfont
                            A}%
                      \vss}%
        }%
        \kern-#2em%
        \TeX}
%    \end{macrocode}
%
% \subsection{Authors, contributors, addresses, signatures}
%
% An article may have several authors (of course), so we permit an
% \cs{author} command for each of them.  The names are then stored in
% a set of \cs{csname}s called \cs{author1}, \cs{author2}, \dots
% Similarly, there are several \cs{address<n>} and \cs{netaddress<n>}
% and \cs{PersonalURL<n>} commands set up for each article.
%
% \begin{description}
% \item[\emph{Comment:}] I would like to make provision for several
%   authors at the same address, but (short of preempting the |*|
%   marker, which it would be nice to retain so as to preserve
%   compatibility with the |plain| style) I'm not sure how one would
%   signal it.
% \end{description}
%    \begin{macrocode}
\def\theauthor#1{\csname theauthor#1\endcsname}
\def\theaddress#1{\csname theaddress#1\endcsname}
\def\thenetaddress#1{\csname thenetaddress#1\endcsname}
\def\thePersonalURL#1{\csname thePersonalURL#1\endcsname}
%    \end{macrocode}
%
% The standard way of listing authors is to iterate from 1 to
% \cs{count@} and to pick the author names as we go.
%
%    \begin{macrocode}
%<!latex>\newcount\@tempcnta
\def\@defaultauthorlist{%
  \@getauthorlist\@firstofone
}
%    \end{macrocode}
%
% \cs{@getauthorlist} processes the author list, passing every bit of
% stuff that needs to be typeset to the macro specified as its argument.
%    \begin{macrocode}
\def\@getauthorlist#1{%
  \count@\authornumber
  \advance\count@ by -2
  \@tempcnta0
%    \end{macrocode}
%
% Loop to output the first \ensuremath{n-2} of the \ensuremath{n}
% authors (the loop does nothing if there are two or fewer authors)
%
%    \begin{macrocode}
  \loop
    \ifnum\count@>0
      \advance\@tempcnta by \@ne
      #1{\ignorespaces\theauthor{\number\@tempcnta}\unskip, }%
      \advance\count@ by \m@ne
  \repeat
  \count@\authornumber
  \advance\count@ by -\@tempcnta
  \ifnum\authornumber>0
%    \end{macrocode}
%
% If there are two or more authors, we output the penultimate author's
% name here, followed by `and'
%
%    \begin{macrocode}
    \ifnum\count@>1
      \count@\authornumber
      \advance\count@ by \m@ne
      #1{\ignorespaces\theauthor{\number\count@}\unskip\ and }%
    \fi
%    \end{macrocode}
%
% Finally (if there were any authors at all) output the last author's
% name:
%
%    \begin{macrocode}
    #1{\ignorespaces\theauthor{\number\authornumber}\unskip}
  \fi
}
%    \end{macrocode}
%
% Signature blocks.  The author can (in principle) define a different
% sort of signature block using \cs{signature}, though this could well
% cause the editorial group to have collective kittens (unless it had
% been discussed in advance\dots)
%
%    \begin{macrocode}
\def\signature#1{\def\@signature{#1}}
\def\@signature{\@defaultsignature}
%    \end{macrocode}
%
% \cs{@defaultsignature} loops through all the authors, outputting the
% details we have about that author, or (if we're in a sub-article)
% outputs the contributor's name and closes the group opened by
% \cs{contributor}.  It is (as its name implies) the default body for
% \cs{makesignature}
%
%    \begin{macrocode}
\def\@defaultsignature{{%
    \let\thanks\@gobble
    \frenchspacing
    %
    \ifnum\authornumber<0
%    \end{macrocode}
% if \cs{authornumber}$<0$, we are in a contributor's section
%    \begin{macrocode}
      \medskip
      \signaturemark
      \theauthor{\number\authornumber}\\
      \theaddress{\number\authornumber}\\
      \allowhyphens
      \thenetaddress{\number\authornumber}\\
      \thePersonalURL{\number\authornumber}\\
    \else
%    \end{macrocode}
% \cs{authornumber}$\ge0$, so we are in the body of an ordinary
% article
%    \begin{macrocode}
      \count@=0
      \loop
        \ifnum\count@<\authornumber
          \medskip
          \advance\count@ by \@ne
          \signaturemark
          \theauthor{\number\count@}\\
          \theaddress{\number\count@}\\
          {%
            \allowhyphens
            \thenetaddress{\number\count@}\\
            \thePersonalURL{\number\count@}\\
          }%
      \repeat
    \fi
  }%
}
\newdimen\signaturewidth   \signaturewidth=12pc
%    \end{macrocode}
% The optional argument to \cs{makesignature} is useful in some
% circumstances (e.g., multi-contributor articles)
%    \begin{macrocode}
\newcommand\makesignature[1][\medskipamount]{%
%    \end{macrocode}
%
% check the value the user has put in \cs{signaturewidth}: it may be
% at most \verb|1.5pc| short of \cs{columnwidth}
%    \begin{macrocode}
  \@tempdima\signaturewidth
  \advance\@tempdima 1.5pc
  \ifdim \@tempdima>\columnwidth
    \signaturewidth \columnwidth
    \advance\signaturewidth -1.5pc
  \fi
  \par
  \penalty9000
  \vspace{#1}%
  \rightline{%
    \vbox{\hsize\signaturewidth \ninepoint \raggedright
      \parindent \z@ \everypar={\hangindent 1pc }
      \parskip \z@skip
      \def\|{\unskip\hfil\break}%
      \def\\{\endgraf}%
      \def\phone{\rm Phone: }
      \rm\@signature}%
  }%
  \ifnum\authornumber<0 \endgroup\fi
}
\def\signaturemark{\leavevmode\llap{$\diamond$\enspace}}
%    \end{macrocode}
%
% The code previously defined the following:
%
% {\small\begin{verbatim}
%   {\makeactive\@
%    \gdef\signatureat{\makeactive\@\def@{\char"40\discretionary{}{}{}}}
%    \makeactive\%
%    \gdef\signaturepercent{\makeactive\%\def%{\char"25\discretionary{}{}{}}}
%   }
% \end{verbatim}}
%
% However, they were never used within the class (or within
% |ltugproc.cls|).  They have therefore been deleted; the identically
% defined \cs{netaddrat} and \cs{netaddrpercent} may be used in the
% unlikely event that they're needed elsewhere.
%
% Now all the awful machinery of author definitions.
% \cs{authornumber} records the number of authors we have recorded to
% date.
%
%    \begin{macrocode}
\newcount\authornumber
\authornumber=0
%    \end{macrocode}
%
% \cs{author} `allocates' another author name (by bumping
% \cs{authornumber}) and also sets up the address and netaddress for
% this author to produce a warning and to prevent oddities if they're
% invoked.  This last assumes that invocation will be in the context
% of \cs{signature} (|ltugboat.cls|) or \cs{maketitle}
% (|ltugproc.cls|); in both cases, invocation is followed by a line
% break (tabular line break |\\| in |ltugproc|, \cs{endgraf} in
% \cs{makesignature} in |ltugboat|).
%
%    \begin{macrocode}
\def\author{%
  \global\advance\authornumber\@ne
  \TB@author
}
%    \end{macrocode}
% \cs{contributor} is for a small part of a multiple-part article; it
% begins a group that will be ended in \cs{makesignature}
%    \begin{macrocode}
\def\contributor{%
  \begingroup
  \authornumber\m@ne
  \TB@author
}
%    \end{macrocode}
%
% Both `types' of author fall through here to set up the author name
% and to initialise author-related things.  \cs{EDITORno*} commands
% allow the editor to record that there's good reason for an
% \emph{address} or \emph{netaddress} not to be there (the
% \emph{personalURL} is optional anyway).
%    \begin{macrocode}
\def\TB@author#1{%
  \expandafter\def\csname theauthor\number\authornumber\endcsname
      {\ignorespaces#1\unskip}%
  \expandafter\def\csname theaddress\number\authornumber\endcsname
    {\TBWarningNL{Address for #1\space missing}\@gobble}%
  \expandafter\def\csname thenetaddress\number\authornumber\endcsname
    {\TBWarningNL{Net address for #1\space missing}\@gobble}%
  \expandafter\let\csname thePersonalURL\number\authornumber\endcsname
    \@gobble
  }
\def\EDITORnoaddress{%
  \expandafter\let\csname theaddress\number\authornumber\endcsname
    \@gobble
}
\def\EDITORnonetaddress{%
  \expandafter\let\csname thenetaddress\number\authornumber\endcsname
    \@gobble
}
%    \end{macrocode}
%
% \cs{address} simply copies its argument into the \cs{theaddress<n>}
% for this author.
%
%    \begin{macrocode}
\def\address#1{%
  \expandafter\def\csname theaddress\number\authornumber\endcsname
      {\leavevmode\ignorespaces#1\unskip}}
%    \end{macrocode}
%
% \cs{network} is for use within the optional argument of
% \cs{netaddress}; it defines the \emph{name} of the network the user
% is on.
%
% \begin{description}
% \item[\emph{Comment:}] I think this is a fantasy, since everyone (in
%   practice, nowadays) quotes an internet address.  In principle, there
%   are people who will quote |X.400| addresses (but they're few and far
%   between) and I have (during 1995!)\ seen an address with an |UUCP|
%   bang-path component on |comp.text.tex|, but \emph{really}!
% \end{description}
%
%    \begin{macrocode}
\def\network#1{\def\@network{#1: }}
%    \end{macrocode}
%
% \cs{netaddress} begins a group, executes an optional argument (which
% should not, presumably, contain global commands) and then relays to
% \cs{@relay@netaddress} with both |@| and |%| made active (so that
% they can be discretionary points in the address).  If we're using
% \LaTeXe, we use the default-argument form of \cs{newcommand};
% otherwise we write it out in all its horribleness.
%
%    \begin{macrocode}
\newcommand\netaddress[1][\relax]{%
  \begingroup
  \def\@network{}%
%    \end{macrocode}
%
% Unfortunately, because of the catcode hackery, we have still to do
% one stage of relaying within our own code, even if we're using
% \LaTeXe.
%
%    \begin{macrocode}
  #1\@sanitize\makespace\ \makeactive\@
  \makeactive\.\makeactive\%\@relay@netaddress}%
%    \end{macrocode}
%
% \cs{@relay@netaddress} finishes the job. It sets \cs{thenetaddress}
% for this author to contain the network name followed by the address.
% As a result of our kerfuffle above, |@| and |%| are active at the
% point we're entered.  We ensure they're active when
% \cs{thenetaddress} gets expanded, too. (\textbf{\emph{WOT}?!})
%
%    \begin{macrocode}
\def\@relay@netaddress#1{%
  \ProtectNetChars
  \expandafter\protected@xdef
      \csname thenetaddress\number\authornumber\endcsname
    {\protect\leavevmode\textrm{\@network}%
     {\protect\NetAddrChars\net
      \ignorespaces#1\unskip}}%
  \endgroup
  }
%    \end{macrocode}
%
% \cs{personalURL} is in essence the same as \cs{netaddress}, apart
% from (1) the lack of the eccentric optional argument, and (2) the
% activation of `|/|'.
%
% For general URLs, \texttt{url.sty} (with or without hyperref) suffices
% and is recommended.
%
%    \begin{macrocode}
\def\personalURL{\begingroup
  \@sanitize\makespace\ \makeactive\@
  \makeactive\.\makeactive\%\makeactive\/\@personalURL}%
\def\@personalURL#1{%
  \ProtectNetChars
  \expandafter\protected@xdef
    \csname thePersonalURL\number\authornumber\endcsname{%
      \protect\leavevmode
      {%
        \protect\URLchars\net
        \ignorespaces#1\unskip
      }%
    }%
  \endgroup
  }
%    \end{macrocode}
%
% Define the activation mechanism for `|@|', `|%|', `|.|'  and `|/|', for use
% in the above.  Note that, since the code has `|%|' active, we have
% `|*|' as a comment character, which has a tendency to make things
% look peculiar\dots
%
%    \begin{macrocode}
{%
  \makecomment\*
  \makeactive\@
  \gdef\netaddrat{\makeactive\@*
    \def@{\discretionary{\char"40}{}{\char"40}}}
  \makeactive\%
  \gdef\netaddrpercent{\makeactive\%*
    \def%{\discretionary{\char"25}{}{\char"25}}}
  \makeactive\.
  \gdef\netaddrdot{\makeactive\.*
    \def.{\discretionary{\char"2E}{}{\char"2E}}}
%    \end{macrocode}
%
% \cs{NetAddrChars} is what \emph{we} use (we're constrained to retain
% the old interface to this stuff, but it \emph{is} clunky\dots).
% Since {\small URL}s are a new idea, we are at liberty not to define
% a separate \cs{netaddrslash} command, and we only have
% \cs{URLchars}.
%
%    \begin{macrocode}
  \gdef\NetAddrChars{\netaddrat \netaddrpercent \netaddrdot}
  \makeactive\/
  \gdef\URLchars{*
    \NetAddrChars
    \makeactive\/*
    \def/{\discretionary{\char"2F}{}{\char"2F}}}
%    \end{macrocode}
%
% \cs{ProtectNetChars} includes protecting `|/|', since this does no
% harm in the case of net addresses (where it's not going to be
% active) and we thereby gain by not having yet another csname.
%
%    \begin{macrocode}
  \gdef\ProtectNetChars{*
    \def@{\protect@}*
    \def%{\protect%}*
    \def.{\protect.}*
    \def/{\protect/}*
    }
}
%    \end{macrocode}
%
%  \LaTeXe{} (in its wisdom) suppresses
%  \cs{DeclareOldFontCommand} when in compatibility mode, so that in
%  that circumstance we need to use a declaration copied from
%  |latex209.def| rather than the way we would normally do the thing
%  (using the command \LaTeXe{} defines for the job).
%
%    \begin{macrocode}
\if@compatibility
  \DeclareRobustCommand\net{\normalfont\ttfamily\mathgroup\symtypewriter}
\else
  \DeclareOldFontCommand{\net}{\ttfamily\upshape\mdseries}{\mathtt}
\fi
\def\authorlist#1{\def\@author{#1}}
\def\@author{\@defaultauthorlist}
%    \end{macrocode}
%
% For the online re-publication (as of 2009) by Mathematical Sciences
% Publishers |http://mathscipub.org|), lots and lots of metadata is
% needed, much of it redundant with things we already do.  They are
% flexible enough to allow us to specify it in any reasonable way, so
% let's make one command \cs{mspmetavar} which takes two arguments.
% Example: |\mspmetavar{volumenumber}{30}|.  For our purposes, it is
% just a no-op.
%
% \begin{macro}{\mspmetavar}
%    \begin{macrocode}
\def\mspmetavar#1#2{}
%    \end{macrocode}
% \end{macro}
%
% \subsection{Article title}
%
% \begin{macro}{\if@articletitle}
% \begin{macro}{\maketitle}
% \begin{macro}{\@r@maketitle}
% \cs{maketitle} takes an optional ``*''; if present, the operation is
% not defining the title of a paper, merely that of a ``business''
% section (such as the participants at a meeting) that has no credited
% author or other title.  In this case, the command flushes out the
% latest \cs{sectitle} (or whatever) but does nothing else.
%
% Provide machinery to skip extra space, even one or more full columns,
% above the top of an article to leave space to paste up a previous
% article that has finished on the same page.  This is a fall back to
% accommodate the fact that multiple articles cannot yet be run together
% easily with \LaTeXe.
%    \begin{macrocode}
\newif\if@articletitle
\def\maketitle{\@ifstar
  {\@articletitlefalse\@r@maketitle}%
  {\@articletitletrue\@r@maketitle}%
}
\def\@r@maketitle{\par
 \ifdim\PreTitleDrop > \z@
   \loop
   \ifdim \PreTitleDrop > \textheight
     \vbox{}\vfil\eject
     \advance\PreTitleDrop by -\textheight
   \repeat
   \vbox to \PreTitleDrop{}
   \global\PreTitleDrop=\z@
 \fi
 \begingroup
 \setcounter{footnote}{0}
 \def\thefootnote{\fnsymbol{footnote}}
 \@maketitle
 \@thanks
 \endgroup
 \setcounter{footnote}{0}
 \gdef\@thanks{}
 }
%    \end{macrocode}
% \end{macro}
% \end{macro}
% \end{macro}
%
% \begin{macro}{\title}
% \begin{macro}{\TB@title}
% We redefine the \cs{title} command, so as to set the \cs{rhTitle}
% command at the same time.  While we're at it, we redefine it to have
% optional arguments for use as `short' versions, thus obviating the
% need for users to use the \cs{shortTitle} command.
%    \begin{macrocode}
\def\rhTitle{}% avoid error if no author or title
\renewcommand\title{\@dblarg\TB@title}
\def\TB@title[#1]#2{\gdef\@title{#2}%
  \bgroup
    \let\thanks\@gobble
    \def\\{\unskip\space\ignorespaces}%
    \protected@xdef\rhTitle{#1}%
  \egroup
}
%    \end{macrocode}
% \end{macro}
% \end{macro}
%
% \begin{macro}{\shortTitle}
% \begin{macro}{\ifshortAuthor}
% \begin{macro}{\shortAuthor}
% The \cs{rh*} commands are versions to be used in the running head of
% the article.  Normally, they are the same things as the author and
% title of the article, but in the case that there are confusions
% therein, the text should provide substitutes, using the \cs{short*}
% commands.
%    \begin{macrocode}
\def\shortTitle #1{\def\rhTitle{#1}}
\newif\ifshortAuthor
\def\shortAuthor #1{\def\rhAuthor{#1}\shortAuthortrue}
%    \end{macrocode}
% \end{macro}
% \end{macro}
% \end{macro}
%
% \subsection{Section titles}
%
%  The following macros are used to set the large \TUB{} section heads
%  (e.g.\ ``General Delivery'', ``Fonts'', etc.)
%
%  Define the distance between articles which are run together:
%
%    \begin{macrocode}
\def\secsep{\vskip 5\baselineskip}
%    \end{macrocode}
%
% Note that \cs{stbaselineskip} is used in the definition of
% \cs{sectitlefont}, in \LaTeXe, so that it has (at least) to be
% defined before \cs{sectitlefont} is used (we do the whole job).
%
%    \begin{macrocode}
\newdimen\stbaselineskip        \stbaselineskip=18\p@
\newdimen\stfontheight
\settoheight{\stfontheight}{\sectitlefont O}
%    \end{macrocode}
%
% Declaring section titles; the conditional \cs{ifSecTitle} records
% the occurence of a \cs{sectitle} command.  If (when) a subsequent
% \cs{maketitle} occurs, the section title box will get flushed out;
% as a result of this, one could in principle have a set of
% \cs{sectitle} commands in a semi-fixed steering file, and inclusions
% of files inserted only as and when papers have appeared.  Only the
% last \cs{sectitle} will actually be executed.
%
%    \begin{macrocode}
\newif\ifSecTitle
\SecTitlefalse
\newif\ifWideSecTitle
\newcommand\sectitle{%
  \SecTitletrue
  \@ifstar
    {\WideSecTitletrue\def\s@ctitle}%
    {\WideSecTitlefalse\def\s@ctitle}%
}
%    \end{macrocode}
%
% \cs{PreTitleDrop} records the amount of column-space we need to
% eject before we start any given paper.  It gets zeroed after that
% ejection has happened.
%
%    \begin{macrocode}
\newdimen\PreTitleDrop   \PreTitleDrop=\z@
%    \end{macrocode}
%
% The other parameters used in \cs{@sectitle}; I don't think there's
% the slightest requirement for them to be registers (since they're
% constant values, AFAIK), but converting them to macros would remove
% the essentially useless functionality of being able to change them
% using assignment, which I'm not about to struggle with just now...
%
% \cs{AboveTitleSkip} and \cs{BelowTitleSkip} are what you'ld expect;
% \cs{strulethickness} is the value to use for \cs{fboxrule} when
% setting the title.
%
%    \begin{macrocode}
\newskip\AboveTitleSkip  \AboveTitleSkip=12\p@
\newskip\BelowTitleSkip  \BelowTitleSkip=8\p@
\newdimen\strulethickness       \strulethickness=.6\p@
%    \end{macrocode}
%
% \cs{@sectitle} actually generates the section title (in a rather
% generous box).  It gets called from \cs{maketitle} under conditional
% \cs{ifSecTitle}; by the time \cs{@sectitle} takes control, we
% already have \cs{SecTitlefalse}.  This implementation uses \LaTeX's
% \cs{framebox} command, on the grounds that one doesn't keep a dog
% and bark for oneself\dots
%
%    \begin{macrocode}
\def\@sectitle #1{%
  \par
  \penalty-1000
%    \end{macrocode}
%
% If we're setting a wide title, the stuff will be at the top of a
% page (let alone a column) but inside a box, so that the separator
% won't be discardable: so don't create the separator in this case.
%
%    \begin{macrocode}
  \ifWideSecTitle\else\secsep\fi
  {%
    \fboxrule\strulethickness
    \fboxsep\z@
    \noindent\framebox[\hsize]{%
      \vbox{%
        \raggedcenter
        \let\\\@sectitle@newline
        \sectitlefont
        \makestrut[2\stfontheight;\z@]%
        #1%
        \makestrut[\z@;\stfontheight]\endgraf
      }%
    }%
  }%
  \nobreak
  \vskip\baselineskip
}
%    \end{macrocode}
%
% \begin{macro}{\@sectitle@newline}
% For use inside \cs{sectitle} as |\\|.  Works similarly to |\\| in
% the ``real world''~--- uses an optional argument
%    \begin{macrocode}
\newcommand{\@sectitle@newline}[1][\z@]{%
  \ifdim#1>\z@
    \makestrut[\z@;#1]%
  \fi
  \unskip\break
}
%    \end{macrocode}
% \end{macro}
%
% We need to trigger the making of a section title in some cases where
% we don't have a section title proper (for example, in material taken
% over from
%^^A \acro
% {TTN}).
%    \begin{macrocode}
\def\@makesectitle{\ifSecTitle
    \global\SecTitlefalse
    \ifWideSecTitle
      \twocolumn[\@sectitle{\s@ctitle}]%
      \global\WideSecTitlefalse
    \else
      \@sectitle{\s@ctitle}%
    \fi
  \else
    \vskip\AboveTitleSkip
    \kern\topskip
    \hrule \@height\z@ \@depth\z@ \@width 10\p@
    \kern-\topskip
    \kern-\strulethickness
    \hrule \@height\strulethickness \@depth\z@
    \kern\medskipamount
    \nobreak
  \fi
}
%    \end{macrocode}
%
% \begin{macro}{\@maketitle}
% Finally, the body of \cs{maketitle} itself.
%    \begin{macrocode}
\def\@maketitle{%
  \@makesectitle
  \if@articletitle{%
    \nohyphens \interlinepenalty\@M
    \setbox0=\hbox{%
      \let\thanks\@gobble
      \let\\=\quad
      \let\and=\quad
      \ignorespaces\@author}%
    {%
      \noindent\bf\raggedright\ignorespaces\@title\endgraf
    }%
    \ifdim \wd0 < 5\p@               % omit if author is null
    \else
%    \end{macrocode}
% Since we have \(\mathtt{\cs{BelowTitleSkip}} + \mathtt{4pt} =
% \mathtt{\cs{baselineskip}}\), we say:
%    \begin{macrocode}
      \nobreak \vskip 4\p@
      {%
        \leftskip=\normalparindent
        \raggedright
        \def\and{\unskip\\}%
        \noindent\@author\endgraf
      }%
    \fi
    \nobreak
    \vskip\BelowTitleSkip
  }\fi%
  \global\@afterindentfalse
  \aftergroup\@afterheading
}
%    \end{macrocode}
% \end{macro}
%
% Dedications are ragged right, in italics.
%
%    \begin{macrocode}
\newenvironment{dedication}%
  {\raggedright\noindent\itshape\ignorespaces}%
  {\endgraf\medskip}
%    \end{macrocode}
%
% The |abstract| and |longabstract| environments both use
% \cs{section*}.
% ^^A [the following is no longer true]
% ^^A They therefore hack at our subversion of
% ^^A \cs{@startsection}, to avoid a confusing warning.
%
%    \begin{macrocode}
\renewenvironment{abstract}%
  {%
    \begin{SafeSection}%
    \section*{Abstract}%
  }%
  {\end{SafeSection}}
\newenvironment{longabstract}%
  {%
    \begin{SafeSection}%
    \section*{Abstract}%
    \bgroup\small
  }%
  {%
    \endgraf\egroup
    \end{SafeSection}%
  \vspace{.25\baselineskip}
  \begin{center}
    {$--*--$}
  \end{center}
  \vspace{.5\baselineskip}}
%    \end{macrocode}
%
% \subsection{Section headings}
%
%  Redefine style of section headings to match plain \TUB.
%  Negative beforeskip suppresses following parindent.  (So
%  negate the stretch and shrink too).
%
%  These macros are called \cs{*head} in the plain styles.
%
%  Relaying via \cs{TB@startsection} detects inappropriate use of
%  \cs{section*}.  Of course, if (when) \emph{we} use it, we need to
%  avoid that relaying; this can be done by \cs{let}ting
%  \cs{TB@startsection} to \cs{TB@safe@startsection}, within a group.
%
% First the version for use in the default case, when class option
% \textsc{numbersec} is in effect.
%
%    \begin{macrocode}
\if@numbersec
  \def\section{\TB@startsection{{section}%
                                1%
                                \z@
                                {-8\p@ \@plus-2\p@ \@minus-2\p@}%
                                {4\p@}%
          {\normalsize\bf\raggedright\hyphenpenalty=\@M}}}
  \def\subsection{\TB@startsection{{subsection}%
                                   2%
                                   \z@
                                   {-8\p@ \@plus-2\p@ \@minus-2\p@}%
                                   {4\p@}%
          {\normalsize\bf\raggedright\hyphenpenalty=\@M}}}
  \def\subsubsection{\TB@startsection{{subsubsection}%
                                      3%
                                      \z@
                                      {-8\p@ \@plus-2\p@ \@minus-2\p@}%
                                      {4\p@}%
          {\normalsize\bf\raggedright\hyphenpenalty=\@M}}}
  \def\paragraph{\TB@startsection{{paragraph}%
                                  4%
                                  \z@
                                  {4\p@ \@plus1\p@ \@minus1\p@}%
                                  {-1em}%
                                  {\normalsize\bf}}}
%    \end{macrocode}
%
% Now the version if class option \textsc{nonumber} is in effect,
% i.e., if \cs{if@numbersec} is false.
%
%    \begin{macrocode}
\else
  \setcounter{secnumdepth}{0}
  \def\section{\TB@nolimelabel
               \TB@startsection{{section}%
                                1%
                                \z@
                                {-8\p@ \@plus-2\p@ \@minus-2\p@}%
                                {4\p@}%
          {\normalsize\bf\raggedright\hyphenpenalty=\@M}}}
  \def\subsection{\TB@nolimelabel
                  \TB@startsection{{subsection}%
                                   2%
                                   \z@
                                   {-8\p@ \@plus-2\p@ \@minus-2\p@}%
                                   {-0.5em\@plus-\fontdimen3\font}%
          {\normalsize\bf\raggedright\hyphenpenalty=\@M}}}
  \def\subsubsection{\TB@nolimelabel
                     \TB@startsection{{subsubsection}%
                                      3%
                                      \parindent
                                      {-8\p@ \@plus-2\p@ \@minus-2\p@}%
                                      {-0.5em\@plus-\fontdimen3\font}%
          {\normalsize\bf\raggedright\hyphenpenalty=\@M}}}
\fi
%    \end{macrocode}
%
% \cs{TB@startsection} traps \verb"*" versions of sectioning
% commands, if numbering isn't in effect.  Its argument is the
% complete set of \cs{@startsection} arguments.
%
%    \begin{macrocode}
\if@numbersec
  \def\TB@startsection#1{\@startsection#1}%
\else
  \def\TB@startsection#1{%
    \@ifstar
      {\TBWarning{*-form of \expandafter\string\csname\@firstofsix#1%
                  \endcsname\space
                  \MessageBreak
                  conflicts with nonumber class option}%
       \@startsection#1}%
      {\@startsection#1}%
  }
\fi
\def\@firstofsix#1#2#3#4#5#6{#1}
%    \end{macrocode}
%
% \cs{TB@safe@startsection} is to be used where \cs{section*} (etc.)\
% appear in places where the request is OK (because it's built in to
% some macro we don't fiddle with).
%
%    \begin{macrocode}
\def\TB@safe@startsection#1{\@startsection#1}
%    \end{macrocode}
%
% The \texttt{SafeSection} environment allows use of \texttt{*}-forms
% of sectioning environments.  It's not documented for the general
% public: it's intended as an editor's facility.
%
%    \begin{macrocode}
\newenvironment{SafeSection}%
  {\let\TB@startsection\TB@safe@startsection}%
  {}
%    \end{macrocode}
%
% And now for the exciting sectioning commands that \LaTeX{} defines
% but we don't have a definition for (whatever else, we don't want
% Lamport's originals, which come out `like the blare of a bugle in a
% lullaby'\footnote{Thurber, \emph{The Wonderful O}}).
%
% The three inappropriate ones are subparagraph (indistinguishable
% from paragraph), and chapter and part.  The last seemed almost to be
% defined in an early version of these macros, since there was a
% definition of \cs{l@part}.  I've not got down to where that came
% from (or why).  If class option \textsc{nonumber} is in effect, we
% also suppress \cs{paragraph}, since it has no parallel in the plain
% style.
%
%    \begin{macrocode}
\if@numbersec
  \def\subparagraph{\TB@nosection\subparagraph\paragraph}
\else
  \def\paragraph{\TB@nosection\paragraph\subsubsection}
  \def\subparagraph{\TB@nosection\subparagraph\subsubsection}
\fi
\def\chapter{\TB@nosection\chapter\section}
\def\part{\TB@nosection\part\section}
\def\TB@nosection#1#2{\TBWarning{class does not support \string#1,
    \string#2\space used instead}#2}
%    \end{macrocode}
%
% \cs{l@<sectioning-name>} is for table of contents (of an article).
%
% We define new macros to allow easily changing the font used for toc
% entries (for \TUB, we usually want roman, not bold), and the space
% between entries.  Nelson Beebe's articles are almost the only ones
% that ever have toc's.
%
%    \begin{macrocode}
\def\TBtocsectionfont{\normalfont}
\newskip\TBtocsectionspace  \TBtocsectionspace=1.0em\@plus\p@
%    \end{macrocode}
%
% Don't ask me (RF) why \cs{l@part} is there; I commented it out
% because I couldn't understand why it had been left there for me.  To
% be finally deleted in a future release of these macros\dots
%    \begin{macrocode}
%\def\l@part#1#2{\addpenalty{\@secpenalty}%
%  \addvspace{2.25em\@plus\p@}%
%  \begingroup
%    \@tempdima 3em \parindent\z@ \rightskip\z@ \parfillskip\z@
%    {\large \bf \leavevmode #1\hfil \hbox to\@pnumwidth{\hss #2}}\par
%    \nobreak
%  \endgroup}
%
\def\l@section#1#2{\addpenalty{\@secpenalty}%
  \addvspace{\TBtocsectionspace}%
  \@tempdima 1.5em
  \begingroup
    \parindent\z@ \rightskip\z@  % article style makes \rightskip > 0
    \parfillskip\z@
    \TBtocsectionfont
    \leavevmode\advance\leftskip\@tempdima\hskip-\leftskip#1\nobreak\hfil
    \nobreak\hb@xt@\@pnumwidth{\hss #2}\par
  \endgroup}
%    \end{macrocode}
%
% \subsection{Appendices}
%
% Appendices (which are really just another sort of section heading)
% raise a problem: if the sections are unnumbered, we plainly need to
% restore the section numbering, which in turn allows labelling of
% section numbers again (\cs{TBnolimelabel} happens before the
% \cs{refstepcounter}, so its effects get lost \dots~what a clever
% piece of design that was).  So here we go:
%
%    \begin{macrocode}
\renewcommand\appendix{\par
  \renewcommand\thesection{\@Alph\c@section}%
  \setcounter{section}{0}%
  \if@numbersec
  \else
    \setcounter{secnumdepth}{1}%
  \fi
%    \end{macrocode}
%
% Now: is this the start of an appendix environment?  This can be
% detected by looking at \cs{@currenvir}; if we are, we need to relay
% to \cs{@appendix@env} to pick up the optional argument.
%
%    \begin{macrocode}
  \def\@tempa{appendix}
  \ifx\@tempa\@currenvir
    \expandafter\@appendix@env
  \fi
}
%    \end{macrocode}
%
% Here we deal with \cs{begin}|{appendix}|\oarg{app-name}
%
%    \begin{macrocode}
\newcommand\app@prefix@section{}
\newcommand\@appendix@env[1][Appendix]{%
  \renewcommand\@seccntformat[1]{\csname app@prefix@##1\endcsname
    \csname the##1\endcsname\quad}%
  \renewcommand\app@prefix@section{#1 }%
}
%    \end{macrocode}
%
% Ending an appendix environment is pretty trivial\dots
%
%    \begin{macrocode}
\let\endappendix\relax
%    \end{macrocode}
%
%
% \subsection{References}
%
% If the sections aren't numbered, the natural tendency of the
% author to cross-reference (which, after all, is one of the things
% \LaTeX{} is for ever being advertised as being good at) can cause
% headaches for the editor.  (Yes it can; believe me~\dots\ there's
% always one.)
%
% The following command is used by each of the sectioning commands to
% make a following \cs{ref} command bloop at the author.  Even if the
% author then ignores the complaint, the poor old editor may find the
% offending \cs{label} rather more easily.
%
% (Note that macro name is to be read as ``\emph{noli me} label'' (I
% don't know the medi\ae val Latin for `label').
%
% \begin{description}
% \item[\emph{Comment}] To come (perhaps): detection of the act of
%   labelling, and an analogue of \cs{ifG@refundefined} for this sort of
%   label
% \end{description}
%
%    \begin{macrocode}
\def\TB@nolimelabel{%
  \def\@currentlabel{%
    \protect\TBWarning{%
      Invalid reference to numbered label on page \thepage
      \MessageBreak made%
    }%
    \textbf{?!?}%
  }%
}
%    \end{macrocode}
%
% \subsection{Title references}
%
% This is a first cut at a mecahnism for referencing by the title of a
% section; it employs the delightfully simple idea Sebastian Rahtz has
% in the \texttt{nameref} package (which is part of
% \texttt{hyperref}).  As it stands, it lacks some of the bells and
% whistles of the original, but they could be added; this is merely
% proof-of-concept.
%
% The name label comes from the moveable bit of the section argument;
% we subvert the \cs{@sect} and \cs{@ssect} commands (the latter
% deals with starred section commands) to grab the relevant argument.
%    \begin{macrocode}
\let\TB@@sect\@sect
\let\TB@@ssect\@ssect
\def\@sect#1#2#3#4#5#6[#7]#8{%
  \def\@currentlabelname{#7}%
  \TB@@sect{#1}{#2}{#3}{#4}{#5}{#6}[{#7}]{#8}%
}
\def\@ssect#1#2#3#4#5{%
 \def\@currentlabelname{#5}%
 \TB@@ssect{#1}{#2}{#3}{#4}{#5}%
}
%    \end{macrocode}
%
% We output the name label as a second \cs{newlabel} command 
% in the \texttt{.aux} file.  That way, packages such as |varioref|
% which also read the \texttt{.aux} information can still work.
% So we redefine \cs{label} to first call the standard \LaTeX\
% \cs{label} and then write our named label as |nr<label>|.
% 
%    \begin{macrocode}
\let\@savelatexlabel=\label  % so save original LaTeX command
%
\def\label#1{% de
  \@savelatexlabel{#1}%
  \@bsphack
  \if@filesw
    \protected@write\@auxout{}%
      {\string\newlabel{nr@#1}{{\@currentlabel}{\@currentlabelname}}}%
  \fi
  \@esphack
}
%    \end{macrocode}
%
% Of course, in the case of a sufficiently mad author, there will be
% no sectioning commands, so we need to
%    \begin{macrocode}
\let\@currentlabelname\@empty
%    \end{macrocode}
%
% Getting named references is then just like getting page references in
% the \LaTeX\ kernel (see |ltxref.dtx|).
%
%    \begin{macrocode}
\DeclareRobustCommand\nameref[1]{\expandafter\@setref
  \csname r@nr@#1\endcsname\@secondoftwo{#1}}
%    \end{macrocode}
%
% \subsection{Float captions}
%
% By analogy with what we've just done to section titles and the like,
% we now do our best to discourage hyphenation within captions.  We also
% typeset them in \cs{small} (actually \cs{tubcaptionfonts}).
%
% First, let's define a dimension by which we will indent full-page
% captions.  We'll also use this to indent abstracts in proceedings style.
% \begin{macro}{\tubfullpageindent}
%    \begin{macrocode}
\newdimen\tubfullpageindent  \tubfullpageindent=4.875pc
%    \end{macrocode}
% \end{macro}
%
% Ok, here is the \cs{@makecaption}.
%    \begin{macrocode}
\def\tubcaptionfonts{\small}%
\long\def\@makecaption#1#2{%
  \vskip\abovecaptionskip
  \sbox\@tempboxa{\tubcaptionfonts \tubmakecaptionbox{#1}{#2}}% try in an hbox
  \ifdim \wd\@tempboxa > \hsize
    {% caption doesn't fit on one line; set as a paragraph.
     \tubcaptionfonts \raggedright \hyphenpenalty=\@M \parindent=1em
     % indent full-width captions {figure*}, but not single-column {figure}.
     \ifdim\hsize = \textwidth
       \leftskip=\tubfullpageindent \rightskip=\leftskip
       \advance\rightskip by 0pt plus2em  % increase acceptable raggedness
     \fi
     \noindent \tubmakecaptionbox{#1}{#2}\par}%
  \else
    % fits on one line; use the hbox, centered.  Do not reset its glue.
    \global\@minipagefalse
    \hb@xt@\hsize{\hfil\box\@tempboxa\hfil}%
  \fi
  \vskip\belowcaptionskip}
% 
\def\tubmakecaptionbox#1#2{#1: #2}% allow overriding for a paper
%    \end{macrocode}
%
% Also use \cs{tubcaptionfonts} for the caption
% labels, and put the label itself (e.g., ``Figure 1'') in bold.
%
%    \begin{macrocode}
\def\fnum@figure{{\tubcaptionfonts \bf \figurename\nobreakspace\thefigure}}
\def\fnum@table{{\tubcaptionfonts \bf \tablename\nobreakspace\thetable}}
%    \end{macrocode}
%
% Let's reduce the default space above captions a bit, and give it some
% flexibility.  The default is |10pt|, which seems too much.
%    \begin{macrocode}
\setlength\abovecaptionskip{6pt plus1pt minus1pt}
%    \end{macrocode}
%
% \subsection{Size changing commands}
%
% Apart from their `normal' effects, these commands change the glue
% around displays.
%    \begin{macrocode}
\renewcommand\normalsize{%
   \@setfontsize\normalsize\@xpt\@xiipt
   \abovedisplayskip=3\p@\@plus 3\p@\@minus\p@
   \belowdisplayskip=\abovedisplayskip
   \abovedisplayshortskip=\z@\@plus 3\p@
   \belowdisplayshortskip=\p@\@plus 3\p@\@minus\p@
}

\renewcommand\small{%
   \@setfontsize\small\@ixpt{11}%
   \abovedisplayskip=2.5\p@\@plus 2.5\p@\@minus\p@
   \belowdisplayskip=\abovedisplayskip
   \abovedisplayshortskip=\z@\@plus 2\p@
   \belowdisplayshortskip=\p@\@plus 2\p@\@minus\p@
}
\renewcommand\footnotesize{%
    \@setfontsize\footnotesize\@viiipt{9.5}%
    \abovedisplayskip=3\p@\@plus 3\p@\@minus\p@
    \belowdisplayskip=\abovedisplayskip
    \abovedisplayshortskip=\z@\@plus 3\p@
    \belowdisplayshortskip=\p@\@plus 3\p@\@minus\p@
}
%    \end{macrocode}
%
% \subsection{Lists and other text inclusions}
%
%    \begin{macrocode}
\def\@listi{%
  \leftmargin\leftmargini\parsep=\p@\@plus\p@\@minus\p@
  \itemsep=\parsep
  \listparindent=1em
  }

\def\@listii{%
  \leftmargin\leftmarginii
  \labelwidth=\leftmarginii \advance\labelwidth-\labelsep
  \topsep=2\p@\@plus\p@\@minus\p@
  \parsep=\p@\@plus\p@\@minus\p@
  \itemsep=\parsep
  \listparindent=1em
  }

\def\@listiii{%
  \leftmargin=\leftmarginiii
  \labelwidth=\leftmarginiii \advance\labelwidth-\labelsep
  \topsep=\p@\@plus\p@\@minus\p@
  \parsep=\z@
  \itemsep=\topsep
  \listparindent=1em
  }
\def\quote{\list{}{\rightmargin.5\leftmargin}\item[]}
%    \end{macrocode}
%
% From Dominik Wujastyk's font article.  First paragraph of a quotation
% will not be indented, and right margin is decreased for narrow columns.
%    \begin{macrocode}
\renewcommand{\quotation}{\list{}{\listparindent 1.5em
     \rightmargin.5\leftmargin\parsep \z@\@plus\p@}\item[]}
%    \end{macrocode}
%
% The |compactitemize| and |compactenumerate| environments, without
% space between the items.
%
%    \begin{macrocode}
\newenvironment{compactitemize}%
   {\begin{itemize}%
     \setlength{\itemsep}{0pt}%
     \setlength{\parskip}{0pt}%
     \setlength{\parsep} {0pt}%
   }%
   {\end{itemize}}
%
\newenvironment{compactenumerate}%
   {\begin{enumerate}%
     \setlength{\itemsep}{0pt}%
     \setlength{\parskip}{0pt}%
     \setlength{\parsep} {0pt}%
   }%
   {\end{enumerate}}
%    \end{macrocode}
%
% \subsection{Some fun with \texttt{verbatim}}
%
% The |plain| \TUB{} style allows \verb|[optional]| arguments to its
% \cs{verbatim} command.  This will allow the author (or editor) to
% specify a range of exciting features; we would definitely like the
% numbered verbatim style for code (that facility is reserved for a
% future version of this package), and the present little bit of code
% imposes the \cs{ruled} option on the built-in \textsf{verbatim}
% environment.  (Note that we don't yet deal with
% \textsf{verbatim}$*$, which is in itself an option to the |plain|
% original.)
%
% We start by saving various bits and bobs whose operation we're going
% to subvert.
%
%    \begin{macrocode}
%\let\@TB@verbatim\@verbatim
\let\@TBverbatim\verbatim
\let\@TBendverbatim\endverbatim
%    \end{macrocode}
%
% Impose an optional argument on the environment.
%
% We start the macro with \cs{par} to avoid a common error: if the
% optional argument is \cs{small}, and the document has no blank line
% before the verbatim block, we don't want that preceding paragraph to
% be set with \cs{small}'s line spacing.
%
% (\cs{obeylines} added to prevent the \cs{futurelet} from propagating
% into the body of the verbatim, thus causing lines that start with
% odd characters (like |#| or even |\|) to behave peculiarly.)
%    \begin{macrocode}
\def\verbatim{\par\obeylines
  \futurelet\reserved@a\@switch@sqbverbatim}
\def\@switch@sqbverbatim{\ifx\reserved@a[%]
  \expandafter\@sqbverbatim\else
  \def\reserved@b{\@sqbverbatim[]}\expandafter\reserved@b\fi}
\def\@sqbverbatim[#1]{%
%    \end{macrocode}
%
% The optional argument consists entirely of functions that modify the
% appearance of the environment.  Following the |plain| style, we
% define the functions we can execute in the optional argument here.
%
% The command \cs{ruled} tells us that there should be rules above and
% below the verbatim block.
%
%    \begin{macrocode}
  \def\ruled{\let\if@ruled\iftrue}%
%    \end{macrocode}
%
% Then we just execute the ones we've got, and relay to a (hacked)
% copy of the built-in environment.
%
%    \begin{macrocode}
  #1\@TBverbatim}
%    \end{macrocode}
%
% The built-in environment itself relays to \cs{@verbatim}, which
% we've subverted to impose our views on appearance.
%
%    \begin{macrocode}
\def\@verbatim{%
%    \end{macrocode}
%
% First, we deal with \cs{ruled}:
%
%    \begin{macrocode}
    \if@ruled\trivlist\item\hrule\kern5\p@\nobreak\fi
%    \end{macrocode}
%
% Now, the code out of the original \textsf{verbatim} environment:
%
%    \begin{macrocode}
  \trivlist \item\relax
  \if@minipage\else\vskip\parskip\fi
  \leftskip\@totalleftmargin\rightskip\z@skip
  \parindent\z@\parfillskip\@flushglue\parskip\z@skip
  \@@par
  \@tempswafalse
  \def\par{%
    \if@tempswa
      \leavevmode \null \@@par\penalty\interlinepenalty
    \else
      \@tempswatrue
      \ifhmode\@@par\penalty\interlinepenalty\fi
    \fi}%
  \obeylines \verbatim@font \@noligs
  \let\do\@makeother \dospecials
  \everypar \expandafter{\the\everypar \unpenalty}%
}%
%    \end{macrocode}
%
% To end the environment, we do everything in reverse order: relay via
% the copy we made of \cs{endverbatim}, and then finish off the option
% changes (again \cs{ruled} only, so far).
%
%    \begin{macrocode}
\def\endverbatim{\@TBendverbatim
  \if@ruled\kern5\p@\hrule\endtrivlist\fi}
%    \end{macrocode}
%
% \cs{enablemetacode} simply typesets\footnote{Or will simply typeset,
%   when we get around to implementation proper} something that looks
% (verbatim) like:
%
% \verb|   <meta-text>|
%
% \noindent as:
%
% \verb|   |\ensuremath{\langle}\textit{meta-text}\ensuremath{\rangle}
%
%    \begin{macrocode}
{\makeactive<
  \gdef<#1>{{\reset@font\ensuremath{\langle}%
    \textit{#1}%
    \ensuremath{\rangle}}}
}
%    \end{macrocode}
%
% Define the \cs{if} used by the \cs{ruled} option:
%
%    \begin{macrocode}
\let\if@ruled\iffalse
%    \end{macrocode}
%
% Finally, if |microtype| is loaded, we want it to be deactivated in
% verbatim blocks.  It often manipulates a leading |\| rather too much.
%    \begin{macrocode}
\AtBeginDocument{%
   \@ifpackageloaded{microtype}
     {\g@addto@macro\@verbatim{\microtypesetup{activate=false}}}{}
}
%    \end{macrocode}

% \subsection{Bibliography}
% \label{sec:citations}
% This is more or less copied verbatim from Glenn Paulley's
% \emph{chicago.sty} (\texttt{gnpaulle@\penalty0 bluebox.\penalty0
% uwaterloo.\penalty0 ca}).
% It produces an author-year citation style bibliography, using
% output from the \BibTeX\ style file based on that by
% Patrick Daly. It needs extra macros beyond those in standard
% \LaTeX\ to function properly.
% The form of the bibitem entries is:
% \begin{verbatim}
%   \bibitem[\protect\citeauthoryear{Jones, Baker, and Smith}
%        {Jones et al.}{1990}{key}...
% \end{verbatim}
% The available citation commands are:
%
% \begin{tabular}{ll}
%    \verb|\cite{key}| & $\rightarrow$     (Jones, Baker, and Smith 1990)\\
%    \verb|\citeA{key}| & $\rightarrow$    (Jones, Baker, and Smith)\\
%    \verb|\citeNP{key}| & $\rightarrow$   Jones, Baker, and Smith 1990\\
%    \verb|\citeANP{key}| & $\rightarrow$  Jones, Baker, and Smith\\
%    \verb|\citeN{key}| & $\rightarrow$    Jones, Baker, and Smith (1990)\\
%    \verb|\shortcite| & $\rightarrow$    (Jones et al.~1990)\\
%    \verb|\citeyear| & $\rightarrow$      (1990)\\
%    \verb|\citeyearNP| & $\rightarrow$    1990\\
% \end{tabular}
%
% First of all (after checking that we're to use Harvard citation at
% all), make a copy of \LaTeX's default citation mechanism.
%
%    \begin{macrocode}
\if@Harvardcite
\let\@internalcite\cite
%    \end{macrocode}
% Normal forms.
%    \begin{macrocode}
\def\cite{\def\@citeseppen{-1000}%
    \def\@cite##1##2{(##1\if@tempswa , ##2\fi)}%
    \def\citeauthoryear##1##2##3{##1, ##3}\@internalcite}
\def\citeNP{\def\@citeseppen{-1000}%
    \def\@cite##1##2{##1\if@tempswa , ##2\fi}%
    \def\citeauthoryear##1##2##3{##1, ##3}\@internalcite}
\def\citeN{\def\@citeseppen{-1000}%
    \def\@cite##1##2{##1\if@tempswa , ##2)\else{)}\fi}%
    \def\citeauthoryear##1##2##3{##1 (##3}\@citedata}
\def\citeA{\def\@citeseppen{-1000}%
    \def\@cite##1##2{(##1\if@tempswa , ##2\fi)}%
    \def\citeauthoryear##1##2##3{##1}\@internalcite}
\def\citeANP{\def\@citeseppen{-1000}%
    \def\@cite##1##2{##1\if@tempswa , ##2\fi}%
    \def\citeauthoryear##1##2##3{##1}\@internalcite}
%    \end{macrocode}
% Abbreviated forms (using \emph{et al.})
%    \begin{macrocode}
\def\shortcite{\def\@citeseppen{-1000}%
    \def\@cite##1##2{(##1\if@tempswa , ##2\fi)}%
    \def\citeauthoryear##1##2##3{##2, ##3}\@internalcite}
\def\shortciteNP{\def\@citeseppen{-1000}%
    \def\@cite##1##2{##1\if@tempswa , ##2\fi}%
    \def\citeauthoryear##1##2##3{##2, ##3}\@internalcite}
\def\shortciteN{\def\@citeseppen{-1000}%
    \def\@cite##1##2{##1\if@tempswa , ##2)\else{)}\fi}%
    \def\citeauthoryear##1##2##3{##2 (##3}\@citedata}
\def\shortciteA{\def\@citeseppen{-1000}%
    \def\@cite##1##2{(##1\if@tempswa , ##2\fi)}%
    \def\citeauthoryear##1##2##3{##2}\@internalcite}
\def\shortciteANP{\def\@citeseppen{-1000}%
    \def\@cite##1##2{##1\if@tempswa , ##2\fi}%
    \def\citeauthoryear##1##2##3{##2}\@internalcite}
%    \end{macrocode}
% When just the  year is needed:
%    \begin{macrocode}
\def\citeyear{\def\@citeseppen{-1000}%
    \def\@cite##1##2{(##1\if@tempswa , ##2\fi)}%
    \def\citeauthoryear##1##2##3{##3}\@citedata}
\def\citeyearNP{\def\@citeseppen{-1000}%
    \def\@cite##1##2{##1\if@tempswa , ##2\fi}%
    \def\citeauthoryear##1##2##3{##3}\@citedata}
%    \end{macrocode}
% Place commas in-between citations in the same |\citeyear|, |\citeyearNP|,
% |\citeN|, or |\shortciteN| command.
% Use something like |\citeN{ref1,ref2,ref3}| and |\citeN{ref4}| for a list.
%
%    \begin{macrocode}
\def\@citedata{%
        \@ifnextchar [{\@tempswatrue\@citedatax}%
                                  {\@tempswafalse\@citedatax[]}%
}

\def\@citedatax[#1]#2{%
\if@filesw\immediate\write\@auxout{\string\citation{#2}}\fi%
  \def\@citea{}\@cite{\@for\@citeb:=#2\do%
    {\@citea\def\@citea{, }\@ifundefined% by Young
       {b@\@citeb}{{\bf ?}%
       \@warning{Citation `\@citeb' on page \thepage \space undefined}}%
{\csname b@\@citeb\endcsname}}}{#1}}%
%    \end{macrocode}
% Don't box citations, separate with ; and a space;
% Make the penalty between citations negative: a good place to break.
%    \begin{macrocode}
\def\@citex[#1]#2{%
\if@filesw\immediate\write\@auxout{\string\citation{#2}}\fi%
  \def\@citea{}\@cite{\@for\@citeb:=#2\do%
    {\@citea\def\@citea{; }\@ifundefined% by Young
       {b@\@citeb}{{\bf ?}%
       \@warning{Citation `\@citeb' on page \thepage \space undefined}}%
{\csname b@\@citeb\endcsname}}}{#1}}%
%    \end{macrocode}
% No labels in the bibliography.
%    \begin{macrocode}
\def\@biblabel#1{}
%    \end{macrocode}
% Set length of hanging indentation for bibliography entries.
%    \begin{macrocode}
\newlength{\bibhang}
\setlength{\bibhang}{2em}
%    \end{macrocode}
% Indent second and subsequent lines of bibliographic entries. Stolen
% from openbib.sty: |\newblock| is set to |{}|.
%    \begin{macrocode}
\newdimen\bibindent
\bibindent=1.5em
\@ifundefined{refname}%
   {\newcommand{\refname}{References}}%
   {}%
%    \end{macrocode}
%
% For safety's sake, suppress the \cs{TB@startsection} warnings
% here\dots
%
%    \begin{macrocode}
\def\thebibliography#1{%
  \let\TB@startsection\TB@safe@startsection
  \section*{\refname
    \@mkboth{\uppercase{\refname}}{\uppercase{\refname}}}%
  \list{[\arabic{enumi}]}{%
    \labelwidth\z@ \labelsep\z@
    \leftmargin\bibindent
    \itemindent -\bibindent
    \listparindent \itemindent
    \parsep \z@
    \usecounter{enumi}}
  \def\newblock{}
  \BibJustification
  \sfcode`\.=1000\relax
}
%    \end{macrocode}
%
% \begin{macro}{etal}
% \begin{macro}{\bibentry}
% Other bibliography odds and ends.
%    \begin{macrocode}
\def\etal{et\,al.\@}
\def\bibentry{%
  \smallskip
  \hangindent=\parindent
  \hangafter=1
  \noindent
  \sloppy
  \clubpenalty500 \widowpenalty500
  \frenchspacing
}
%    \end{macrocode}
% \end{macro}
% \end{macro}
%
% \begin{macro}{\bibliography}
% \begin{macro}{\bibliographystyle}
% Changes made to accommodate TUB file naming conventions
%    \begin{macrocode}
\def\bibliography#1{%
  \if@filesw
    \immediate\write\@auxout{\string\bibdata{\@tubfilename{#1}}}%
  \fi
  \@input{\jobname.bbl}%
}
\def\bibliographystyle#1{%
  \if@filesw
    \immediate\write\@auxout{\string\bibstyle{\@tubfilename{#1}}}%
  \fi
}
%    \end{macrocode}
% \end{macro}
% \end{macro}
%
% \begin{macro}{\thebibliography}
% \begin{macro}{\TB@@thebibliography}
% If the user's asked to use \LaTeX's default citation mechanism
% (using the |rawcite| option), we still need to play with
% \cs{TB@startsection}: this is a boring fact of life\dots
%
% We also patch \cs{sloppy} in case there's a need for alternative
% justification of the body of the bibliography.
%    \begin{macrocode}
\else
\let\TB@@thebibliography\thebibliography
\def\thebibliography{%
  \let\TB@startsection\TB@safe@startsection
  \let\sloppy\BibJustification
  \TB@@thebibliography}
\fi
%    \end{macrocode}
% \end{macro}
% \end{macro}
%
% \begin{macro}{\BibJustification}
% \begin{macro}{\SetBibJustification}
% \begin{macro}{\TB@@sloppy}
% \cs{BibJustification} defines how the bibliography is to be
% justified.  The Lamport default is simply ``\cs{sloppy}'', but we
% regularly find some sort of ragged right setting is appropriate.
% (\cs{BibJustification} is nevertheless reset to its default value at
% the start of a paper.)
%    \begin{macrocode}
\let\TB@@sloppy\sloppy
\let\BibJustification\TB@@sloppy
\newcommand{\SetBibJustification}[1]{%
  \renewcommand{\BibJustification}{#1}%
}
\ResetCommands\expandafter{\the\ResetCommands
  \let\BibJustification\TB@@sloppy
}
%    \end{macrocode}
% \end{macro}
% \end{macro}
% \end{macro}
%
% \subsection{Registration marks}
%
% We no longer use these since Cadmus does not want them.
%    \begin{macrocode}
\def\HorzR@gisterRule{\vrule \@height 0.2\p@ \@depth\z@ \@width 0.5in }
\def\DownShortR@gisterRule{\vrule \@height 0.2\p@ \@depth 1pc \@width 0.2\p@ }
\def\UpShortR@gisterRule{\vrule \@height 1pc \@depth\z@ \@width 0.2\p@ }
%    \end{macrocode}
%
% ``T'' marks centered on top and bottom edges of paper
%
%    \begin{macrocode}
\def\ttopregister{\dlap{%
        \hb@xt@\trimwd{\HorzR@gisterRule \hfil \HorzR@gisterRule
                        \HorzR@gisterRule \hfil \HorzR@gisterRule}%
        \hb@xt@\trimwd{\hfil \DownShortR@gisterRule \hfil}}}
\def\tbotregister{\ulap{%
        \hb@xt@\trimwd{\hfil \UpShortR@gisterRule \hfil}%
        \hb@xt@\trimwd{\HorzR@gisterRule \hfil \HorzR@gisterRule
                        \HorzR@gisterRule \hfil \HorzR@gisterRule}}}
\def\topregister{\ttopregister}
\def\botregister{\tbotregister}
%    \end{macrocode}
%
% \subsection{Running heads}
%
%    \begin{macrocode}
\def \rtitlex{\def\texttub##1{{\normalsize\textrm{##1}}}\TUB, \volx }
\def\PrelimDraftfooter{%
  \dlap{\kern\textheight\kern3pc
        \rlap{\hb@xt@\pagewd{\midrtitle\hfil\midrtitle}}
  }}
%    \end{macrocode}
% registration marks; these are temporarily inserted in the running head
%    \begin{macrocode}
\def\MakeRegistrationMarks{}
\def\UseTrimMarks{%
  \def\MakeRegistrationMarks{%
    \ulap{\rlap{%
       \vbox{\dlap{\vbox to\trimlgt{\vfil\botregister}}%
             \topregister\vskip \headmargin \vskip 10\p@}}}}%
  }
% put issue identification and page number in header.
\def\@oddhead{\MakeRegistrationMarks\PrelimDraftfooter
  \normalsize\csname normalshape\endcsname\rm \tubheadhook
  \rtitlex\qquad\midrtitle \hfil \thepage}
\def\@evenhead{\MakeRegistrationMarks\PrelimDraftfooter
  \normalsize\csname normalshape\endcsname\rm \tubheadhook
  \thepage\hfil\midrtitle\qquad\rtitlex}

% can be used to reset the font, e.g., tb98kuester.
\def\tubheadhook{}

% put title and author in footer.
\def\@tubrunningfull{%
  \def\@oddfoot{% make line break commands produce a normal space
    \def\\{\unskip\ \ignorespaces}%
    \let\newline=\\%
    \hfil\rhTitle}
  \def\@evenfoot{\@author\hfil}
}

\def\@tubrunninggetauthor#1{#1
  \begingroup
    \let\thanks\@gobble
    \protected@xdef\rhAuthor{\the\toks@##1}%
  \endgroup
}%

% empty footer.
\def\@tubrunningminimal{%
  \def\@oddfoot{\hfil}%
  \def\@evenfoot{\hfil}%
}

% empty footer and header.
\def\@tubrunningoff{%
  \def\@oddfoot{\hfil}%
  \def\@evenfoot{\hfil}%
  \def\@oddhead{\hfil}%
  \def\@evenhead{\hfil}%
}

\def\ps@headings{}
\pagestyle{headings}
%    \end{macrocode}
%
% \subsection{Output routine}
% Modified to alter |\brokenpenalty| across columns
%
% \begin{description}
% \item[\emph{Comment}] We're playing with fire here: for example,
%   \cs{@outputdblcol} has changed in \LaTeXe{} for 1995/06/01 (with the
%   use of \cs{hb@xt@}).  \emph{This} time there's no semantic change,
%   but\dots
% \end{description}
%    \begin{macrocode}
\def\@outputdblcol{\if@firstcolumn \global\@firstcolumnfalse
    \global\setbox\@leftcolumn\box\@outputbox
    \global\brokenpenalty10000
  \else \global\@firstcolumntrue
    \global\brokenpenalty100
    \setbox\@outputbox\vbox{\hb@xt@\textwidth{\hb@xt@\columnwidth
      {\box\@leftcolumn \hss}\hfil \vrule \@width\columnseprule\hfil
       \hb@xt@\columnwidth{\box\@outputbox \hss}}}\@combinedblfloats
       \@outputpage \begingroup \@dblfloatplacement \@startdblcolumn
       \@whilesw\if@fcolmade \fi{\@outputpage\@startdblcolumn}\endgroup
    \fi}
%    \end{macrocode}
%
% \subsection{Font-related definitions and machinery}
% These are mostly for compatibility with plain |tugboat.sty|
%    \begin{macrocode}
\newif\ifFirstPar       \FirstParfalse
\def\smc{\sc}
\def\ninepoint{\small}
%</classtail>
%    \end{macrocode}
%
% \cs{SMC} \emph{isn't} small caps~--- Barbara Beeton says she thinks
% of it as ``big small caps''.  She says (modulo capitalisation of
% things\dots):
% \begin{quote}
%   For the things it's used for, regular small caps are not
%   appropriate~--- they're too small.  Real small caps are
%   appropriate for author names (and are so used in continental
%   bibliographies), section headings, running heads, and, on
%   occasion, words to which some emphasis is to be given.  \cs{SMC}
%   was designed to be used for acronyms and all-caps abbreviations,
%   which look terrible in small caps, but nearly as bad in all caps
%   in the regular text size.  The principle of using ``one size
%   smaller'' than the text size is similar to the design of caps in
%   German~--- where they are smaller relative to lowercase than are
%   caps in fonts intended for English, to improve the appearance of
%   regular text in which caps are used at the heads of all nouns, not
%   just at the beginnings of sentences.
% \end{quote}
%
% We define this in terms of the memory of the size currently selected
% that's maintained in \cs{@currsize}: if the user does something
% silly re.~selecting fonts, we'll get the wrong results.  The
% following code is adapted from an old version of |relsize.sty| by
% Donald Arseneau and Matt Swift.  (The order of examination of
% \cs{@currsize} is to get the commonest cases out of the way first.)
%    \begin{macrocode}
%<*common>
\DeclareRobustCommand\SMC{%
  \ifx\@currsize\normalsize\small\else
   \ifx\@currsize\small\footnotesize\else
    \ifx\@currsize\footnotesize\scriptsize\else
     \ifx\@currsize\large\normalsize\else
      \ifx\@currsize\Large\large\else
       \ifx\@currsize\LARGE\Large\else
        \ifx\@currsize\scriptsize\tiny\else
         \ifx\@currsize\tiny\tiny\else
          \ifx\@currsize\huge\LARGE\else
           \ifx\@currsize\Huge\huge\else
            \small\SMC@unknown@warning
 \fi\fi\fi\fi\fi\fi\fi\fi\fi\fi
}
\newcommand\SMC@unknown@warning{\TBWarning{\string\SMC: nonstandard
    text font size command -- using \string\small}}
\newcommand\textSMC[1]{{\SMC #1}}
%    \end{macrocode}
%
% The \cs{acro} command uses \cs{SMC} as it was originally intended.
% Note that, since most of these things are uppercase-only names, it
% fiddles with the spacefactor after inserting its text.
%
%    \begin{macrocode}
\newcommand\acro[1]{\textSMC{#1}\@}
%</common>
%    \end{macrocode}
%
% \subsection{Miscellaneous definitions}
%
% \cs{EdNote} allows the editor to enter notes in the text of a paper.
% If the command is given something that appears like an optional
% argument, the entire text of the note is placed in square brackets.
% (Yes, it really is!)
%
%    \begin{macrocode}
%<*classtail>
\def\xEdNote{{\EdNoteFont Editor's note:\enspace }}
\def \EdNote{\@ifnextchar[%]
  {%
    \ifvmode
      \smallskip\noindent\let\@EdNote@\@EdNote@v
    \else
      \unskip\quad\def\@EdNote@{\unskip\quad}%
    \fi
    \@EdNote
  }%
  \xEdNote
}
\long\def\@EdNote[#1]{%
  [\thinspace\xEdNote\ignorespaces
   #1%
   \unskip\thinspace]%
  \@EdNote@
}
\def\@EdNote@v{\par\smallskip}
%    \end{macrocode}
%  Macros for Mittelbach's self-documenting style
%    \begin{macrocode}
\def\SelfDocumenting{%
  \setlength\textwidth{31pc}
  \onecolumn
  \parindent \z@
  \parskip 2\p@\@plus\p@\@minus\p@
  \oddsidemargin 8pc
  \evensidemargin 8pc
  \marginparwidth 8pc
  \toks@\expandafter{\@oddhead}%
  \xdef\@oddhead{\hss\hb@xt@\pagewd{\the\toks@}}%
  \toks@\expandafter{\@evenhead}%
  \xdef\@evenhead{\hss\hb@xt@\pagewd{\the\toks@}}%
  \def\ps@titlepage{}%
}
\def\ps@titlepage{}

\long\def\@makefntext#1{\parindent 1em\noindent\hb@xt@2em{}%
  \llap{\@makefnmark}\null$\mskip5mu$#1}

%% \long\def\@makefntext#1{\parindent 1em
%%   \noindent
%%   \hb@xt@2em{\hss\@makefnmark}%
%%   \hskip0.27778\fontdimen6\textfont\z@\relax
%%   #1%
%% }
%    \end{macrocode}
%
% \begin{macro}{\creditfootnote}
% \begin{macro}{\supportfootnote}
% Sometimes we want the label ``Editor's Note:'', sometimes not.
%
%    \begin{macrocode}
\def\creditfootnote{\nomarkfootnote\xEdNote}
\def\supportfootnote{\nomarkfootnote\relax}
%    \end{macrocode}
%
% General macro \cs{nomarkfootnote} to make a footnote without a
% reference mark, etc.  |#1| is an extra command to insert, |#2| the
% user's text.
%    \begin{macrocode}
\gdef\nomarkfootnote#1#2{\begingroup
  \def\thefootnote{}%
  % no period, please, also no fnmark.
  \def\@makefntext##1{##1}%
  \footnotetext{\noindent #1#2}%
  \endgroup
}
%    \end{macrocode}
% \end{macro}
% \end{macro}
%
%
% \subsection{Initialization}
%
% If we're going to use Harvard-style bibliographies, we set up the
% bibliography style: the user doesn't get any choice.
%
%    \begin{macrocode}
\if@Harvardcite
  \AtBeginDocument{%
    \bibliographystyle{ltugbib}%
  }
\fi
\authornumber\z@
\let\@signature\@defaultsignature
\InputIfFileExists{ltugboat.cfg}{\TBInfo{Loading ltugboat
                                           configuration information}}{}
%</classtail>
%    \end{macrocode}
%
% \section{\protect\LaTeXe\ Proceedings class}
%
% \begin{macro}{\@tugclass}
% Make the code of |ltugboat.cls| (when we load it) say it's really
% us:
%
%    \begin{macrocode}
%<*ltugproccls>
\def\@tugclass{ltugproc}
%    \end{macrocode}
% \end{macro}
%
% \begin{macro}{\if@proctw@column}
% For the case where we're preparing the preprints, we may not have
% been able to prepare submissions for typesetting in two columns.  In
% this case, therefore, we may need the option |onecolumn|, that will
% suppress the use of twocolumn setting within the article.
%
%    \begin{macrocode}
\newif\if@proctw@column  \@proctw@columntrue
\DeclareOption{onecolumn}{\@proctw@columnfalse}
%    \end{macrocode}
% \end{macro}
%
% \begin{macro}{\if@proc@sober}
% \begin{macro}{\if@proc@numerable}
% ^^A\acro
% {TUG}'96 proceedings switched to more sober headings still; so
% the \textsf{tug95} option establishes the original state.  In the
% absence of any other guidance, we use the '96 for
% ^^A\acro
% {TUG}'97 proceedings, but also allow numbering of sections.
%
%    \begin{macrocode}
\newif\if@proc@sober
\newif\if@proc@numerable
\DeclareOption{tug95}{%
  \@proc@soberfalse
  \@proc@numerablefalse
}
\DeclareOption{tug96}{%
  \@proc@sobertrue
  \@proc@numerablefalse
}
\DeclareOption{tug97}{%
  \@proc@sobertrue
  \@proc@numerabletrue
}
\DeclareOption{tug2002}{%
  \@proc@sobertrue
  \@proc@numerabletrue
  \let\if@proc@numbersec\iftrue
  \PassOptionsToClass{numbersec}{ltugboat}%
}
%    \end{macrocode}
% \end{macro}
% \end{macro}
%
% \begin{macro}{\if@proc@numbersec}
% If we're in a class that allows section numbering (the actual check
% occurs after \cs{ProcessOptions}, we can have the
% following:
%    \begin{macrocode}
\DeclareOption{numbersec}{\let\if@proc@numbersec\iftrue
  \PassOptionsToClass{numbersec}{ltugboat}%
}
\DeclareOption{nonumber}{\let\if@proc@numbersec\iffalse
  \PassOptionsToClass{nonumber}{ltugboat}%
}
%    \end{macrocode}
% \end{macro}
%
% \begin{macro}{\ifTB@title}
% If we have a paper for which we want to create a detached title,
% with an editor's note, and then set the paper separately, we use
% option notitle.
%    \begin{macrocode}
\newif\ifTB@title
\DeclareOption{title}{\TB@titletrue}
\DeclareOption{notitle}{\TB@titlefalse
  \AtBeginDocument{\stepcounter{page}}}
%    \end{macrocode}
% \end{macro}
%
% There are these people who seem to think |tugproc| is an option as
% well as a class\dots
%
%    \begin{macrocode}
\DeclareOption{tugproc}{%
  \ClassWarning{\@tugclass}{Option \CurrentOption\space ignored}%
}
%    \end{macrocode}
%
% All other options are simply passed to |ltugboat|\dots
%
%    \begin{macrocode}
\DeclareOption*{\PassOptionsToClass{\CurrentOption}{ltugboat}}
%    \end{macrocode}
%
% If there's a \textsf{tugproc} defaults file, input it now: it may
% tell us which year we're to perform for\dots  (Note: this code
% \emph{is} millenium-proof.  It's not terribly classy for years
% beyond 2069, but then I'm not going to be around then---this will be
% an interesting task for a future \TeX{}ie\dots)
%
%    \begin{macrocode}
\InputIfFileExists{\@tugclass.cfg}{\ClassInfo{ltugproc}%
           {Loading ltugproc configuration information}}{}
\@ifundefined{TUGprocExtraOptions}%
   {\let\TUGprocExtraOptions\@empty}%
   {\edef\TUGprocExtraOptions{,\TUGprocExtraOptions}}
%    \end{macrocode}
%
% \begin{macro}{\tugProcYear}
% Now work out what year it is
%    \begin{macrocode}
\@tempcnta\year
\ifnum\@tempcnta<2000
  \divide\@tempcnta by100
  \multiply\@tempcnta by100
  \advance\@tempcnta-\year
  \@tempcnta-\@tempcnta
\fi
%    \end{macrocode}
%
% And use that for calculating a year for us to use.
%
%    \begin{macrocode}
\edef\@tempa{\noexpand\providecommand\noexpand\tugProcYear
                {\ifnum10>\@tempcnta0\fi\the\@tempcnta}}
\@tempa
\ClassInfo{ltugproc}{Class believes year is
  \expandafter\ifnum\tugProcYear<2000 19\fi\tugProcYear
    \@gobble}
%    \end{macrocode}
%
% Check that this is a ``sensible year'' (one for which we have a
% class option defined).  If not, make it a `suitable' year, in
% particular, one that allows numbering sections.
%    \begin{macrocode}
\expandafter\ifx\csname ds@tug\tugProcYear\endcsname\relax
  \def\tugProcYear{2002}\fi
%    \end{macrocode}
% \end{macro}
%
% Now execute the default `year' option and get on with processing.
% Note that this command gets ignored if the configuration file
% specifies a silly year.
%
%    \begin{macrocode}
\ExecuteOptions{tug\tugProcYear,title\TUGprocExtraOptions}
\ProcessOptions
\if@proc@numbersec
  \if@proc@numerable
  \else
    \ClassWarning{\@tugclass}{This year's proceedings may not have
      numbered sections}%
  \fi
\fi
%    \end{macrocode}
%
% Call \textsf{ltugboat}, adding whichever section numbering option is
% appropriate
%    \begin{macrocode}
\LoadClass[\if@proc@numbersec numbersec\else nonumber\fi]{ltugboat}
%    \end{macrocode}
%
% \subsection{Proceedings titles}
%
% \begin{macro}{\maketitle}
% \begin{macro}{\ifTB@madetitle}
% There's no provision for `section titles' in proceedings issues, as
% there are in \TUB{} proper.  Note the tedious \LaTeX{} bug-avoidance
% in the \cs{@TB@test@document} macro.
%
%    \begin{macrocode}
\def\maketitle{%
  \begingroup
%    \end{macrocode}
%
% first, a bit of flim-flam to generate an initial value for
% \cs{rhAuthor} (unless the user's already given one with a
% \cs{shortAuthor} comand).
%    \begin{macrocode}
    \ifshortAuthor\else
      \global\let\rhAuthor\@empty
      \def\g@addto@rhAuthor##1{%
        \begingroup
          \toks@\expandafter{\rhAuthor}%
          \let\thanks\@gobble
          \protected@xdef\rhAuthor{\the\toks@##1}%
        \endgroup
      }%
      \@getauthorlist\g@addto@rhAuthor
    \fi
%    \end{macrocode}
%
% now, the real business of setting the title
%    \begin{macrocode}
    \ifTB@title
      \setcounter{footnote}{0}%
      \renewcommand\thefootnote{\@fnsymbol\c@footnote}%
      \if@proctw@column
        \twocolumn[\@maketitle]%
      \else
        \onecolumn
        \global\@topnum\z@
        \@maketitle
      \fi
      \@thanks
      \thispagestyle{TBproctitle}
    \fi
  \endgroup
  \TB@madetitletrue
}
\newif\ifTB@madetitle  \TB@madetitlefalse
%    \end{macrocode}
% \end{macro}
% \end{macro}
%
% \begin{macro}{\@TB@test@document}
% \cs{@TB@test@document} checks to see, at entry to \cs{maketitle}, if
% we've had \verb:\begin{document}:.  See \LaTeX{} bug report latex/2212,
% submitted by Robin Fairbairns, for details.
%    \begin{macrocode}
\def\@TB@test@document{%
  \edef\@tempa{\the\everypar}
  \def \@tempb{\@nodocument}
  \ifx \@tempa\@tempb
    \@nodocument
  \fi
}
%    \end{macrocode}
% \end{macro}
%
% \begin{macro}{\AUTHORfont}
% \begin{macro}{\TITLEfont}
% \begin{macro}{\addressfont}
% \begin{macro}{\netaddrfont}
% Define the fonts for titles and things
%    \begin{macrocode}
\def\AUTHORfont {\large\rmfamily\mdseries\upshape}
\def\TITLEfont  {\Large\rmfamily\mdseries\upshape}
\def\addressfont{\small\rmfamily\mdseries\upshape}
\def\netaddrfont{\small\ttfamily\mdseries\upshape}
%    \end{macrocode}
% \end{macro}
% \end{macro}
% \end{macro}
% \end{macro}
%
% \begin{macro}{\aboveauthorskip}
% \begin{macro}{\belowauthorskip}
% \begin{macro}{\belowabstractskip}
% Some changeable skips to permit variability in page layout depending
% on the particular paper's page breaks.
%    \begin{macrocode}
\newskip\aboveauthorskip   \aboveauthorskip=18\p@ \@plus4\p@
\newskip\belowauthorskip   \belowauthorskip=\aboveauthorskip
\newskip\belowabstractskip \belowabstractskip=14\p@ \@plus3\p@ \@minus2\p@
%    \end{macrocode}
% \end{macro}
% \end{macro}
% \end{macro}
%
% \begin{macro}{\@maketitle}
% The body of \cs{maketitle}
%    \begin{macrocode}
\def\@maketitle{%
   {\parskip\z@
    \frenchspacing
    \TITLEfont\raggedright\noindent\@title\par
      \count@=0
      \loop
      \ifnum\count@<\authornumber
        \vskip\aboveauthorskip
        \advance\count@\@ne
        {\AUTHORfont\theauthor{\number\count@}\endgraf}%
        \addressfont\theaddress{\number\count@}\endgraf
        {%
          \allowhyphens
          \hangindent1.5pc
          \netaddrfont\thenetaddress{\number\count@}\endgraf
          \hangindent1.5pc
          \thePersonalURL{\number\count@}\endgraf
        }%
      \repeat
   \vskip\belowauthorskip}%
   \if@abstract
      \centerline{\bfseries Abstract}%
      \vskip.5\baselineskip\rmfamily
      \list{}{\listparindent20\p@
         \itemindent\z@ \leftmargin\tubfullpageindent
         \rightmargin\leftmargin \parsep \z@}\item[]\ignorespaces
            \the\abstract@toks
      \endlist\global\@ignoretrue
   \fi
   \vskip\belowabstractskip
   \global\@afterindentfalse\aftergroup\@afterheading
  }
%    \end{macrocode}
% \end{macro}
%
% \begin{environment}{abstract}
% \begin{macro}{\if@abstract}
% \begin{macro}{\abstract@toks}
% Save the contents of the abstract environment in the token register
% \cs{abstract@toks}.  We need to do this, as otherwise it may get
% `typeset' (previously, it got put in a box) before
% |\begin{document}|, and experiments prove that this means our shiny
% new \cs{SMC} doesn't work in this situation.
%
% If you need to understand the ins and outs of this code, look at the
% place I lifted it from: |tabularx.dtx| (in the tools
% bundle).  The whole thing pivots on having stored the name of the
% `abstract' environment in \cs{@abstract@}
%
%    \begin{macrocode}
\newtoks\abstract@toks  \abstract@toks{}
\let\if@abstract\iffalse
\def\abstract{%
%    \end{macrocode}
%
% we now warn unsuspecting users who provide an \texttt{abstract}
% environment \emph{after} the \cs{maketitle} that would typeset it\dots
%    \begin{macrocode}
  \ifTB@madetitle
    \TBWarning{abstract environment after \string\maketitle}
  \fi
  \def\@abstract@{abstract}%
  \ifx\@currenvir\@abstract@
  \else
    \TBError{\string\abstract\space is illegal:%
      \MessageBreak
      use \string\begin{\@abstract@} instead}%
      {\@abstract@\space may only be used as an environment}
  \fi
  \global\let\if@abstract\iftrue
  {\ifnum0=`}\fi
  \@abstract@getbody}
\let\endabstract\relax
%    \end{macrocode}
%
% \cs{@abstract@getbody} gets chunks of the body (up to the next
% occurrence of \cs{end}) and appends them to \cs{abstract@toks}.  It
% then uses \cs{@abstract@findend} to detect whether this \cs{end} is
% followed by \verb|{abstract}|
%
%    \begin{macrocode}
\long\def\@abstract@getbody#1\end{%
  \global\abstract@toks\expandafter{\the\abstract@toks#1}%
  \@abstract@findend}
%    \end{macrocode}
%
% Here we've got to \cs{end} in the body of the abstract.
% \cs{@abstract@findend} takes the `argument' of the \cs{end} do its
% argument.
%
%    \begin{macrocode}
\def\@abstract@findend#1{%
  \def\@tempa{#1}%
%    \end{macrocode}
%
% If we've found an `end' to match the `begin' that we started with,
% we're done with gathering the abstract up; otherwise we stuff the
% end itself into the token register and carry on.
%
%    \begin{macrocode}
  \ifx\@tempa\@abstract@
    \expandafter\@abstract@end
  \else
%    \end{macrocode}
%
% It's not \cs{end}|{abstract}| --- check that it's not
% \cs{end}|{document}| either (which signifies that the author's
% forgotten about ending the abstract)
%    \begin{macrocode}
    \def\@tempb{document}%
    \ifx\@tempa\@tempb
      \TBError{\string\begin{\@abstract@}
          ended by \string\end{\@tempb}}%
        {You've forgotten \string\end{\@abstract@}}
    \else
       \global\abstract@toks\expandafter{\the\abstract@toks\end{#1}}%
       \expandafter\expandafter\expandafter\@abstract@getbody
    \fi
  \fi}
%    \end{macrocode}
%
% In our case, the action at the `proper' \cs{end} is a lot simpler
% than what appears in |tabularx.dtx| \dots{} don't be surprised!
%
%    \begin{macrocode}
\def\@abstract@end{\ifnum0=`{\fi}%
  \expandafter\end\expandafter{\@abstract@}}
%    \end{macrocode}
% \end{macro}
% \end{macro}
% \end{environment}
%
% \begin{macro}{\makesignature}
% \cs{makesignature} is improper in proceedings, so we replace it with a
% warning (and a no-op otherwise)
%    \begin{macrocode}
\renewcommand{\makesignature}{\TBWarning
         {\string\makesignature\space is invalid in proceedings issues}}
%    \end{macrocode}
% \end{macro}
%
% \begin{macro}{\ps@TBproctitle}
% \begin{macro}{\ps@TBproc}
% \begin{macro}{\dopagecommands}
% \begin{macro}{\setpagecommands}
% \begin{macro}{\TB@definefeet}
% \begin{macro}{\pfoottext}
% \begin{macro}{\rfoottext}
% Now we define the running heads in terms of the \cs{rh*} commands.
%    \begin{macrocode}
\def\ps@TBproctitle{\let\@oddhead\MakeRegistrationMarks
  \let\@evenhead\MakeRegistrationMarks
  \TB@definefeet
}
\def\ps@TBproc{%
  \def\@oddhead{\MakeRegistrationMarks
    {%
      \hfil
      \def\\{\unskip\ \ignorespaces}%
      \rmfamily\rhTitle
    }%
  }%
  \def\@evenhead{\MakeRegistrationMarks
    {%
      \def\\{\unskip\ \ignorespaces}%
      \rmfamily\rhAuthor
      \hfil
    }%
  }%
  \TB@definefeet
}

\advance\footskip8\p@    % for deeper running feet

\def\dopagecommands{\csname @@pagecommands\number\c@page\endcsname}
\def\setpagecommands#1#2{\expandafter\def\csname @@pagecommands#1\endcsname
  {#2}}
\def\TB@definefeet{%
  \def\@oddfoot{\ifpreprint\pfoottext\hfil\Now\hfil\thepage
    \else\rfoottext\hfil\thepage\fi\dopagecommands}%
  \def\@evenfoot{\ifpreprint\thepage\hfil\Now\hfil\pfoottext
    \else\thepage\hfil\rfoottext\fi\dopagecommands}%
}

\def\pfoottext{{\smc Preprint}:
   Proceedings of the \volyr{} Annual Meeting}
\def\rfoottext{\normalfont\TUB, \volx\Dash
   {Proceedings of the \volyr{} Annual Meeting}}

\pagestyle{TBproc}
%    \end{macrocode}
% \end{macro}
% \end{macro}
% \end{macro}
% \end{macro}
% \end{macro}
% \end{macro}
% \end{macro}
%
% \subsection{Section divisions}
%
% Neither sections nor subsections are numbered by default in the
% proceedings style: note that this puts a degree of stress on authors'
% natural tendency to reference sections, which is a matter that needs
% attention.  The class option \textsc{numbersec} once again numbers the
% sections (and noticeably changes the layout).
%
%    \begin{macrocode}
\if@proc@numbersec
\else
  \setcounter{secnumdepth}{0}
\fi
%    \end{macrocode}
%
% Otherwise, the \cs{section} command is pretty straightforward.
% However, the \cs{subsection} and \cs{subsubsection} are run-in, and
% we have to remember to have negative stretch (and shrink if we
% should in future choose to have one) on the \meta{afterskip}
% parameter of \cs{@startsection}, since the whole skip is going to
% end up getting negated.  We use \cs{TB@startsection} to detect
% inappropriate forms.
%
%    \begin{macrocode}
\if@proc@numbersec
\else
  \if@proc@sober
    \def\section
           {\TB@nolimelabel
            \TB@startsection{{section}%
                             1%
                             \z@%
                             {-8\p@\@plus-2\p@\@minus-2\p@}%
                             {6\p@}%
                             {\normalsize\bfseries\raggedright}}}
  \else
    \def\section
           {\TB@nolimelabel
            \TB@startsection{{section}%
                             1%
                             \z@%
                             {-8\p@\@plus-2\p@\@minus-2\p@}%
                             {6\p@}%
                             {\large\bfseries\raggedright}}}
  \fi
  \def\subsection
           {\TB@nolimelabel
            \TB@startsection{{subsection}%
                             2%
                             \z@%
                             {6\p@\@plus 2\p@\@minus2\p@}%
                             {-5\p@\@plus -\fontdimen3\the\font}%
                             {\normalsize\bfseries}}}
  \def\subsubsection
           {\TB@nolimelabel
            \TB@startsection{{subsubsection}%
                             3%
                             \parindent%
                             \z@%
                             {-5\p@\@plus -\fontdimen3\the\font}%
                             {\normalsize\bfseries}}}
\fi
%</ltugproccls>
%    \end{macrocode}
%
% \section{Plain \TeX\ styles}
%    \begin{macrocode}
%<*tugboatsty>
% err...
%</tugboatsty>
%<*tugprocsty>
% err...
%</tugprocsty>
%    \end{macrocode}
% \section{The \LaTeXe\ compatibility-mode style files}
%    \begin{macrocode}
%<*ltugboatsty>
\@obsoletefile{ltugboat.cls}{ltugboat.sty}
\LoadClass{ltugboat}
%</ltugboatsty>
%<*ltugprocsty>
\@obsoletefile{ltugproc.cls}{ltugproc.sty}
\LoadClass{ltugproc}
%</ltugprocsty>
%    \end{macrocode}
%
% ^^A \addtolength{\premulticols}{5\baselineskip}
% \clearpage
%
% \Finale
%
\endinput
%
%% \CharacterTable
%%  {Upper-case    \A\B\C\D\E\F\G\H\I\J\K\L\M\N\O\P\Q\R\S\T\U\V\W\X\Y\Z
%%   Lower-case    \a\b\c\d\e\f\g\h\i\j\k\l\m\n\o\p\q\r\s\t\u\v\w\x\y\z
%%   Digits        \0\1\2\3\4\5\6\7\8\9
%%   Exclamation   \!     Double quote  \"     Hash (number) \#
%%   Dollar        \$     Percent       \%     Ampersand     \&
%%   Acute accent  \'     Left paren    \(     Right paren   \)
%%   Asterisk      \*     Plus          \+     Comma         \,
%%   Minus         \-     Point         \.     Solidus       \/
%%   Colon         \:     Semicolon     \;     Less than     \<
%%   Equals        \=     Greater than  \>     Question mark \?
%%   Commercial at \@     Left bracket  \[     Backslash     \\
%%   Right bracket \]     Circumflex    \^     Underscore    \_
%%   Grave accent  \`     Left brace    \{     Vertical bar  \|
%%   Right brace   \}     Tilde         \~}
